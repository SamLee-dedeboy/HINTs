\section{Introduction}
% summartization -> overview / text mining / transforming large collection of documents into a compact representation / visual summarization
% compact, interactive summary, support analysis, interpretation and trust
% existing methods: based on word co-occurrence -> inaccurate, not able to capture complex relations, not easy to reason why relation exists
% interactions supported: semantic zooming, filtering, highlighting, linking
% analysis supported: global (network statistics, communities) vs. local (paths, motifs)
% 
In text analysis, providing a compact overview of a large collection of documents is a common yet challenging task.
A large collection of documents, or a corpus, could contain multiple interesting characteristics, while the analyst may not already know what to expect.
State-of-the-art techniques can be mainly divided into two categories: word clouds based and topic models based, each having its own advantages and disadvantages.
Word clouds are easy to understand and can be generated quickly, but they are not able to capture the relationships between words and support for further analytical tasks~\cite{viegas2008timelines}.
Topic models can capture the relationships between words, but they are not easy to understand and require parameter tuning~\cite{chuang2013topicdiagnostic}.
The ability to model complex relations lie within the text and interpretability are two important factors to consider when providing a compact overview of a corpus, and they are often in conflict with each other.
Word clouds and topic models lie on the two extremes: word clouds are easy to understand but too simple to capture complex relations, and topic models are able to capture complex relations but not easy to understand.

In addition, a common assumption shared among text analysis methods (not limited to word clouds and topic models) is that each document is a bag of words, and words appearing near to each other indicate meaningful relations.
There are two main problems with this assumption.
First, the fact that meaningful relations between words are informed solely by their co-occurring frequency indicates that the relation is unknown.
Co-occurring frequency merely serves as a signal for the possibility that a relation exists.
Many visual analytics system did not clearly convey this weak signaling. 
Instead, many systems assumes the extracted relations already have meanings.
They then visualize the extracted relations for users to interpret the exact meanings they convey.
This misinformed visualization thus leads to misinterpretation~\cite{lee2017human}.
Second problem is that the meaning hidden under co-occurrence frequency either does not exist at all, or is not to the user's interest in most cases~\cite{EESurveyBiomed}.
For example, Chun et al.~\cite{chun2006extraction} found in a biomedical text corpus that only 30\% of protein pairs co-occurring in the same sentences have an actual interconnection.
When looking for a specific relation, co-occurrence can not provide much help, and thus any methods that build upon co-occurrence will inevitably fail.

The proposed method, event hyper graph, is able to capture complex relations while being easy to understand.
It utilizes the power of event extraction models to capture the complex relations in the text, and uses hyper graphs to visualize the result and support further analytical tasks.
The extracted relations are expressed as \textit{events} rather than co-occurrences,
and the semantic meaning of the relations are explicitly represented by the event through its type, trigger and arguments, which are all human-readable text mentions.
By organizing events into a hyper graph, we provide a compact overview of the corpus in a graph where events and words are interconnected. 
The graph form representation allows users to interact with the overview through common interactions such as filtering, zooming and highlighting.
It also opens the door to incorporate hyper network analysis methods into text analysis, such as community detection and motif finding.
A study by Antelmi et al.~\cite{antelmi2020analyzing} found that in certain scenarios where relations could exist between more than two nodes, conducting analysis on hyper networks can provide more accurate results than on regular networks.
In the case of event hyper graphs,
the nodes are all meaningful words (entities) connected by events, the resulting hyper graph contains diverse relations extracted from the corpus.
Conducting hyper network analysis on such a network is something that has not been explored by the literature.
The next step of this project is to find appropriate case studies to employ those hyper network analysis methods, and investigate the most appropriate interactions to support. 

Our contributions are:
\begin{itemize}
    \item a novel method that combines event extraction models and hyper graphs to provide a compact, interactive overview of a large collection of documents.
    \item a system that combines the proposed overview with hyper network analysis methods for user to explore and analyze a large collection of documents.
    \item two case studies that demonstrate the effectiveness, interpretability and trustworthiness of our method.
\end{itemize}

