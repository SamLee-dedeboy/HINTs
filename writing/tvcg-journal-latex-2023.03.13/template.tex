% \documentclass[journal]{vgtc}                     % final (journal style)
%\documentclass[journal,hideappendix]{vgtc}        % final (journal style) without appendices
\documentclass[review,journal]{vgtc}              % review (journal style)
% \documentclass[review,journal,hideappendix]{vgtc} % review (journal style)
%\documentclass[widereview]{vgtc}                  % wide-spaced review
%\documentclass[preprint,journal]{vgtc}            % preprint (journal style)


%% Uncomment one of the lines above depending on where your paper is
%% in the conference process. ``review'' and ``widereview'' are for review
%% submission, ``preprint'' is for pre-publication in an open access repository,
%% and the final version doesn't use a specific qualifier.

%% If you are submitting a paper to a conference for review with a double
%% blind reviewing process, please use one of the ``review'' options and replace the value ``0'' below with your
%% OnlineID. Otherwise, you may safely leave it at ``0''.
\onlineid{0}

%% In preprint mode you may define your own headline. If not, the default IEEE copyright message will appear in preprint mode.
%\preprinttext{To appear in IEEE Transactions on Visualization and Computer Graphics.}

%% In preprint mode, this adds a link to the version of the paper on IEEEXplore
%% Uncomment this line when you produce a preprint version of the article 
%% after the article receives a DOI for the paper from IEEE
%\ieeedoi{xx.xxxx/TVCG.201x.xxxxxxx}

%% declare the category of your paper, only shown in review mode
\vgtccategory{Research}

%% please declare the paper type of your paper to help reviewers, only shown in review mode
%% choices:
%% * algorithm/technique
%% * application/design study
%% * evaluation
%% * system
%% * theory/model
\vgtcpapertype{application/design study}

%% Paper title.
\title{HyperMap: Analyzing large collections of documents with hypergraphs}

%% Author ORCID IDs should be specified using \authororcid like below inside
%% of the \author command. ORCID IDs can be registered at https://orcid.org/.
%% Include only the 16-digit dashed ID.
\author{
    \authororcid{Sam Yu-Te Lee}{0009-0000-2629-3954}, and
    \authororcid{Kwan-Liu Ma}{0000-0001-8086-0366}
}
\authorfooter{
 \item Sam Yu-Te Lee, and Kwan-Liu Ma are with the University of California, Davis. E-mail: 
\{ ytlee, klma\}@ucdavis.edu
}

%% Abstract section.
\abstract{%
Sensemaking on large collections of documents (corpus) is a challenging task that analysts often have to perform.
Previous works approach this problem either from a topic- or entity-based perspective, but they lack interpretability and trust due to poor model alignment.
In this paper, we propose HyperMap, a visual analytics approach that combines topic- and entity-based techniques seamlessly.
By leveraging the capability of Large Language Models (LLMs), we model the corpus as a hypergraph that matches the user's mental model when analyzing a corpus.
The hypergraph is then hierarchically clustered with an agglomerative clustering algorithm by combining semantic and connectivity similarity.
The system is designed to emphasize Model Alignment to foster interpretability and trust.
To demonstrate the generalizability and effectiveness of the HyperMap system, we present two case studies on two different datasets: a news article dataset and a visualization publication dataset.
We discuss limitations and future work of combining visualization and LLMs to enhance analysts' ability to analyze a corpus.
}

%% Keywords that describe your work. Will show as 'Index Terms' in journal
%% please capitalize first letter and insert punctuation after last keyword
\keywords{Text visualization, Sensemaking, Hypergraph, Corpus analysis Large language models}

%% A teaser figure can be included as follows
% \teaser{
%   \centering
%   \includegraphics[width=\linewidth]{CypressView}
%   \caption{%
%   	In the Clouds: Vancouver from Cypress Mountain.
%   	Note that the teaser may not be wider than the abstract block.%
%   }
%   \label{fig:teaser}
% }
\teaser{
  \centering
  \includegraphics[height=8cm, keepaspectratio]{teaser}
  \caption{The HyperMap system. 
  (a) The peripheral area of Cluster View shows the mentioned characters of highlighted documents using Gilbert curves.
  (b) The center area of Cluster View shows the topic structure of the corpus using Gosper curves.
  (c) The Document View shows a list of selected documents.
  (d) The Chatbot View provides a chatbot interface to answer user questions with the option to insert selected documents in the prompt.}
\label{fig: sfc}
}

%% Uncomment below to disable the manuscript note
%\renewcommand{\manuscriptnotetxt}{}

%% Copyright space is enabled by default as required by guidelines.
%% It is disabled by the 'review' option or via the following command:
%\nocopyrightspace


%%%%%%%%%%%%%%%%%%%%%%%%%%%%%%%%%%%%%%%%%%%%%%%%%%%%%%%%%%%%%%%%
%%%%%%%%%%%%%%%%%%%%%% LOAD PACKAGES %%%%%%%%%%%%%%%%%%%%%%%%%%%
%%%%%%%%%%%%%%%%%%%%%%%%%%%%%%%%%%%%%%%%%%%%%%%%%%%%%%%%%%%%%%%%

%% Tell graphicx where to find files for figures when calling \includegraphics.
%% Note that due to the \DeclareGraphicsExtensions{} call it is no longer necessary
%% to provide the the path and extension of a graphics file:
%% \includegraphics{diamondrule} is completely sufficient.
\graphicspath{{figs/}{figures/}{pictures/}{images/}{./}} % where to search for the images

%% Only used in the template examples. You can remove these lines.
\usepackage{tabu}                      % only used for the table example
\usepackage{booktabs}                  % only used for the table example
\usepackage{lipsum}                    % used to generate placeholder text
\usepackage{mwe}                       % used to generate placeholder figures
\usepackage{microtype}                 % use micro-typography (slightly more compact, better to read)
% \PassOptionsToPackage{warn}{textcomp}  % to address font issues with \textrightarrow
\usepackage{textcomp}                  % use better special symbols
\usepackage{times}                     % we use Times as the main font
\renewcommand*\ttdefault{txtt}         % a nicer typewriter font
\usepackage{cite}                      % needed to automatically sort the references
% \usepackage{amsmath}
\usepackage{subfig}
% \usepackage{algpseudocode}
\usepackage{algorithm}
% \usepackage{enumitem}
% \usepackage{mathptmx}                  % use matching math font
% \usepackage[dvipsnames]{xcolor}
% \definecolor{ride_hailing_technology}{RGB}{242, 142, 44}
% \definecolor{middle_east_threats}{RGB}{89, 161, 79}
% \definecolor{controversies_and_challenges}{RGB}{78, 121, 167}
% \definecolor{criminal_justice}{RGB}{153, 86, 51}
% \definecolor{political_protests}{RGB}{130, 48, 207}
% \definecolor{concerns_and_controversies}{RGB}{131, 230, 77}
% \definecolor{presidential_election_and_controversies}{RGB}{48, 202, 129}
% \definecolor{presidential_eligibility}{RGB}{240, 98, 130}
% \definecolor{ben_carson}{RGB}{187, 187, 187}
% \definecolor{vis_web_data}{RGB}{240, 129, 55}
\begin{document}


%%%%%%%%%%%%%%%%%%%%%%%%%%%%%%%%%%%%%%%%%%%%%%%%%%%%%%%%%%%%%%%%
%%%%%%%%%%%%%%%%%%%%%% START OF THE PAPER %%%%%%%%%%%%%%%%%%%%%%
%%%%%%%%%%%%%%%%%%%%%%%%%%%%%%%%%%%%%%%%%%%%%%%%%%%%%%%%%%%%%%%%

%% The ``\maketitle'' command must be the first command after the
%% ``\begin{document}'' command. It prepares and prints the title block.
%% the only exception to this rule is the \firstsection command


\maketitle
\section{Introduction}
Text data is ubiquitous.
From news articles and social media posts to scientific publications, the tremendous amount of text data that is produced poses not only opportunities but also a great challenge to anyone who needs to analyze them.
Visual analytics (VA) mitigates this challenge by combining mathematical models and visualizations to automate the sensemaking process and reduce the cognitive load.
Chuang et al.~\cite{chuang2012interpretation} proposed that \textit{Model alignment}, the alignment of analysis tasks, visual encodings and model decisions,
greatly affects users' interpretation and trust in visual analytic systems.
However, in text analysis, the available models often align poorly with analysis tasks.
For example, topic models are commonly used to model the topical structure of text documents.
Most topic models characterize \textit{topic} as a probabilistic distribution spanning a given vocabulary~\cite{vayansky2020review}.
This transformation from \textit{topics}, a high-level concept that the user seeks to understand, to a \textit{probabilistic distribution}, a low-level concept that mathematical models can operate on, prevents proper model alignment.
The misalignment between analysis tasks and models limits the usage of visual analytics systems for users who are not familiar with the underlying models.

Recent advances in large language models (LLM) present a promising solution to this problem.
LLMs have proven successful in various natural language processing (NLP) tasks, especially in question-answering tasks due to their strong capability to understand user intent.
Researchers in visualization have adopted LLMs to assist data transformation~\cite{wang2023dataformulator} or directly generate visualization~\cite{maddigan2023chat2vis}.
However, they all assumed a clean data format, where the data to be visualized is already in a table format. 
For unstructured text analysis though, this is rarely the case.
Topics~\cite{atzberger2023evaluatetopicmodel}, sentiments~\cite{beasley2021through}, concepts and entities~\cite{park2018conceptvector,cao2010facetatlas} are common analysis targets in text analysis, which require a data preparation stage to extract them from unstructured text.
Recently, Li et al.~\cite{li2023evaluateChatgpt} evaluated ChatGPT's capabilities on Information Extraction (IE) tasks comprehensively, and found that it excels under an OpenIE setting, where the model relies solely on user input to extract information from documents.
The capability of LLMs to extract information from documents according to user intent eliminates the need to carefully align the analysis tasks and models in VA systems,
because a specific model is no longer needed to prepare the data for the analysis task.
In the previous example, instead of relying on abstruse and unfathomable probabilistic models, LLMs can directly process the text data and summarize the topics of the documents.
A user can ask a LLM:\@ \textit{`What are the topics of these articles?'}, and the LLM would give a human-like response, such as \textit{`The articles are about \ldots'}.

However, using LLMs in the data preparation stage is not trivial. 
Problems like \textit{hallucination} and \textit{faithfulness} hinder the accuracy of the extracted information.
Token limits restrict the length of the input text, limiting the usage of LLMs on large collections of documents (corpus).
Prompts need to be carefully designed to reflect user intent.
Finally, the extracted information, the analysis task and the visualization need to be aligned to foster interpretation and trust.
In this work, we designed a VA system that models a corpus as a hypergraph, where the nodes are articles and salient entities (or concepts) extracted from the articles.
We showcase how LLMs are used flexibly to align the data, analysis task and visualization during our design process.
The hypergraph is then hierarchically clustered and visualized by extending space-filling curve layouts~\cite{muelder2008sfc}.
The system supports interactive exploration, reorganization and analysis of the documents.
To the best of our knowledge, no visual analytics system has adopted LLMs to assist the data preparation stage in text analysis.
Using the system, we demonstrate how proper model alignment can be achieved using LLMs.

The contributions of our work are as follows:
\begin{itemize}
    \item We introduce an LLM-based information extraction pipeline that is capable of extracting topics and salient entities from a given corpus in a way that fosters interpretation.
    \item We extend space-filling curve layouts to visualize clusters in large hypergraphs.
    \item We develop a novel VA system that allows users to effectively explore, reorganize and analyze a corpus.
\end{itemize}











\section{Related Works}
\subsection{Hypergraph visualization}
\cite{fischer2021hypergraphsurvey}
We decided to use node-link-based representation because it is more familiar to users~\cite{abdelaal2022network}.
\subsection{Summarizing large collections of text}
Topic models, entity-based summarization (VA approaches)
\subsection{Interpretability and Trust in text analysis}


% \subsection{Design Rationale}~\label{sec: design_rationale}
HyperMap is designed for analysts to explore and reorganize a corpus for their analysis.
Our design rationale to foster interpretation is based on the design guidelines proposed by Chuang et al.~\cite{chuang2012interpretation}.
We reuse their definitions of \textit{Model Alignment}, \textit{Progressive Disclosure}, and \textit{Unit of Analysis} when describing our design rationale.
We first identify common analysis tasks from previous works.
Then, we derive our design considerations (DC) from the analysis tasks.
We take the DCs into account when making our model decisions and visualization design in~\autoref{sec: methodology}.
Finally, we explain how we achieve model alignment by applying the design considerations to our system.

\subsubsection{Analysis Tasks}
We derive our target analysis tasks from topic- and entity-based approaches.
Topic-based approaches aim to support document understanding by visualizing the topic structure of the documents.
Investigation of the topic structure seeks to answer the question: \textit{What topics are discussed in the corpus, and how are they related?}
Entity-based approaches support investigative analysis by visualizing entities and their relations.
Similarly, the investigation of the entities seeks to answer the question: \textit{What entities are involved in the corpus, and how are they related?}: 
We aim to support both tasks simultaneously as they are fundamental to subsequent tasks and intertwined in a real-world scenario.

\subsubsection{Design Considerations}
To support the aforementioned analysis tasks, we derive the following design considerations (DCs):
\begin{itemize}
  \item \textbf{DC1: Overview of topic structures and entity connections}
  Given a corpus, the topic structures and entity connections can be complex and cover a wide range of articles and entities. 
  The overview seeks to cover all the articles and entities by hiding the details.
  This sets the ground for the user to discover their targets of interest.
  \item \textbf{DC2: Progressive Disclosure}
  To facilitate investigation, it is important to support users to drill down from a high-level overview to intermediate abstractions.
  This includes disclosure of a specific topic's sub-structure, the containing articles, and the connections to entities.
  \item \textbf{DC3: Model Alignment}
  Our choice of model should align well with the analyst's mental model when conducting the analysis tasks.
  This means our model should directly operate on the units of analysis, which are topics (groups of similar articles) and entities.
  Then by properly visualizing the abstractions of the model, we are safe to produce a good model alignment.
  \item \textbf{DC4: Detailed analysis of the target of interest}
  The investigation of topic structures and entity connections often leads to a target of interest, which can be a topic or an entity.
  After such investigation, previous works usually only provide the user with a list of documents that are relevant to the target of interest.
  This is perhaps due to the lack of a unified way to analyze the target of interest under different contexts.
  The advance of LLMs presents a promising solution to this problem by transforming almost any analysis task into a question-answering task.
  We thus include this task to fill the gap in previous works.
\end{itemize}






% \section{Methodology}
\subsection{Modeling}
\subsubsection{Hypergraph Construction}
A hypergraph is a generalization of a graph in which an edge can connect any number of nodes~\cite{fischer2021hypergraphsurvey}.
A hyperedge thus represents a multi-way relationship between nodes.
In this paper, we model two types of hypergraphs: article hypergraph and participant hypergraph, where articles and participants are the nodes, respectively.
\textit{Participants} are the core components that the article's content discuss~\cite{use_other_works_to_refine_definition}.
For example, in a news article, the participants can be named entities such as people, organizations, or locations.
In a research article, the participants can be the concepts or techniques used in the article.

Conducting analysis on article hypergraph and participant hypergraph correspond to topic-based and entity-based analysis, respectively.
Following the definition of a hypergraph node, a hyperedge can be used to represent two types of multi-way relationships:
(1) A hyperedge between \textit{articles} can be constructed if the articles all mention the same participant. 
In this case, the hyperedge represents the co-mention of a participant, i.e.\ a named entity or a concept;
(2) A hyperedge between \textit{participants} can be constructed if the participants are mentioned together in the same article.
In this case, the hyperedge represents a co-occurrence relationship between participants.
Once the two hypergraphs are constructed, they are hierarchically clustered separately.
Clusters in the article hypergraph represents topics that are discussed in the dataset.
Clusters in the participant hypergraph represents participants (entities or concepts) that frequently co-occurred in an article.
For better interpretability of the clustering result, we further assign \textit{tags} for each cluster, which is further explained in~\autoref{sec: tag_assignment}.

Although these two types of hyperedges are constructed differently, we utilize the \textit{dual} of a hypergraph to simplify the construction process.
The dual of a hypergraph is simply another hypergraph, where the hyperedges are now nodes and the nodes are now hyperedges. (Add formulas here to explain).
Therefore, we first model the articles as nodes and participants as hyperedges to construct the article hypergraph $H_A$.
The participant hypergraph $H_P$ can then be easily constructed by taking the dual of $H_A$.
This construction process also allows us to use the same clustering algorithm on both hypergraphs, which is further explained in~\autoref{sec: clustering}.

\subsubsection{Hierarchical Clustering}\label{sec: clustering}
Common clustering algorithms for graphs consider only graph connectivity.
However, for the best interpretability of the clustering result, the node embeddings must be also used in the clustering process.
The necessity of incorporating node embeddings is further explained in~\autoref{sec: tag_assignment}.
Therefore, this limits our choice of clustering algorithms to attributed node clustering algorithms.

Although there are existing approaches that can cluster attributed nodes on graphs such as EVA~\cite{citraro2020eva} and iLouvain~\cite{combe2015louvain}, they are not designed for hypergraphs.
In general, hypergraphs can be clustered in two different ways: 
(1) Directly operate on the hyperedges by generalizing the graph clustering algorithms.
For example, Kamiński et.al.~\cite{kaminski2021hgraphcommunity} generalizes the modularity metric for graphs to hypergraphs; 
(2) First transform the hypergraph into a graph and then apply normal graph clustering algorithms~\cite{kumar2020new}.
Although the first approach is more intuitive, it is less scalable and hard to incorporate node attributes.
Thus, we decided to design our clustering algorithm following the second approach.

Considering all the above, we implemented our hierarchical clustering algorithm by first transforming the hypergraph into a graph following the edge re-weighting process proposed by Kumar et.al.~\cite{kumar2020new},
then an agglomerative clustering algorithm~\cite{steinbach2000doccluster} is applied on the re-weighted graph.
In agglomerative clustering, the key is to define the similarity between nodes and similarity between clusters.
We can easily incorporate node attributes into the clustering process by defining the similarity between nodes and clusters as a combination of attribute similarity $S_s$ and connectivity similarity $S_c$.
Since we're dealing with texts, we refer to the attribute similarity between nodes as semantic similarity. 

The semantic similarity $S_s(i, j)$ is the cosine similarity of the embeddings of the two nodes, denoted as $v_i$.
For article nodes, the embeddings are generated using the article content.
For participant nodes, the embeddings are generated using a description note of the participant.
More details about the embeddings are explained in~\autoref{sec: embeddings}.
The connectivity similarity $S_c$ is the weighted Topological Overlap (wTO)~\cite{gysi2018wto},
which is a weighted generalization of the Overlap Coefficient~\cite{vijaymeena2016survey}, as shown in~\autoref{eq:connectivity_similarity}.
\begin{equation}\label{eq:connectivity_similarity}
    S_s(i, j) = \frac{v_i \cdot v_j}{||v_i|| \cdot ||v_j||}, \quad
    S_c(i, j) = \frac{\sum_{u=1}^N{w_{i,u}w_{u_j}} + w_{i,j}}{\min(k_i, k_j) + 1 - |w_{i,j}|}
\end{equation}
where $k_i = \sum_{j=1}^N |w_{i,j}|$ is the total weight of the edges connected to node $i$.
Finally, a weighting factor $\alpha$ is used to balance the two similarities, as shown in~\autoref{eq: similarity}.
\begin{equation}\label{eq: similarity}
    S = \alpha S_s + (1-\alpha) S_c
\end{equation}
For the similarity between clusters, we used centroid similarity, i.e.\ the similarity between two clusters is the similarity between the centroids of the two clusters.
The algorithm is presented in (TODO: add algorithm pseudocode here)



\subsection{Preprocessing}
The Methodology can work for any unstructured dataset
\subsubsection{Summarization}
Chatgpt for summarization
\subsubsection{Document Embedding}\label{sec: embeddings}
OpenAI's embedding API
\subsubsection{Participant Extraction}\label{sec: participant_extraction}
Chatgpt for major participant extraction and another model for entity linking

\subsubsection{Topic Assignment}\label{sec: tag_assignment}
Chatgpt to assign topics to each cluster


% \section{Visualization}
\subsection{Space Filling Curves}
Introduce Gosper curve and generalized Hilbert curve, and how they are used for large graph layout

\subsection{SFC for HyperGraph}
Using the Gosper curve to layout the article graph

Concatenating four generalized Hilbert curve to layout the entity graph on the peripheral
\subsection{Spacing Strategy}

\subsection{Border Approximation}

\subsection{Edge Bundling}


% \section{System Design}
\subsection{Cluster View}
Use SFC hypergraph to show topical structure of the dataset.

\subsubsection{Interactions}
\subparagraph{Click}
\subparagraph{Expansion}
\subparagraph{Filtering}
\subparagraph{Searching}

\subsection{Article View}

\subsection{Analysis View}
% \section{Evaluation}\label{sec: evaluation}
\subsection{Case Study}
\subsubsection{AllTheNews Dataset}
\subsubsection{Vis Publication Dataset}
\input{sections/08_Limitations_and_Future_Work.tex}
% \vspace*{1cm}
\section{Conclusion}
In this paper, we propose HyperMap, a visual analytics system that assists users in exploring, reorganizing and analyzing a corpus.
Previous works have shown that topic- and entity-based analysis are essential to sensemaking on a corpus,
but existing text analysis tools do not provide a unified interface for both types of analysis.
We fill in this gap by combining state-of-the-art large language models and hypergraph analysis and visualization techniques.
We introduce an LLM-based pipeline to extract topics and characters from unstructured text documents.
We model the corpus as hypergraphs and apply an agglomerative clustering algorithm.
The clusters are visualized by building upon existing space-filling curve layouts, which exhibit a high level of visual scalability and aesthetics.
Moreover, we emphasize the importance of Model Alignment in the design of visual analytics systems.
The generalizability and effectiveness of HyperMap are demonstrated in two case studies, analyzing datasets from two different domains.
Our future research aims to utilize the capabilities of LLMs to understand user intent to support more advanced analysis tasks.






%% if specified like this the section will be ommitted in review mode
% \acknowledgments{%
% 	The authors wish to thank A, B, and C.
%   This work was supported in part by a grant from XYZ (\# 12345-67890).%
% }


% \bibliographystyle{abbrv-doi-hyperref}
%\bibliographystyle{abbrv-doi-hyperref-narrow}
\bibliographystyle{abbrv-doi}
%\bibliographystyle{abbrv-doi-narrow}

\bibliography{template}


\appendix % You can use the `hideappendix` class option to skip everything after \appendix

\end{document}

