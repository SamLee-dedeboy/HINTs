% $Id: template.tex 11 2007-04-03 22:25:53Z jpeltier $

% \documentclass{vgtc}                          % final (conference style)
\documentclass[review]{vgtc}                 % review
%\documentclass[widereview]{vgtc}             % wide-spaced review
%\documentclass[preprint]{vgtc}               % preprint
%\documentclass[electronic]{vgtc}             % electronic version

%% Uncomment one of the lines above depending on where your paper is
%% in the conference process. ``review'' and ``widereview'' are for review
%% submission, ``preprint'' is for pre-publication, and the final version
%% doesn't use a specific qualifier. Further, ``electronic'' includes
%% hyperreferences for more convenient online viewing.

%% Please use one of the ``review'' options in combination with the
%% assigned online id (see below) ONLY if your paper uses a double blind
%% review process. Some conferences, like IEEE Vis and InfoVis, have NOT
%% in the past.

%% Figures should be in CMYK or Grey scale format, otherwise, colour 
%% shifting may occur during the printing process.

%% These few lines make a distinction between latex and pdflatex calls and they
%% bring in essential packages for graphics and font handling.
%% Note that due to the \DeclareGraphicsExtensions{} call it is no longer necessary
%% to provide the the path and extension of a graphics file:
%% \includegraphics{diamondrule} is completely sufficient.
%%
\ifpdf%                                % if we use pdflatex
  \pdfoutput=1\relax                   % create PDFs from pdfLaTeX
  \pdfcompresslevel=9                  % PDF Compression
  \pdfoptionpdfminorversion=7          % create PDF 1.7
  \ExecuteOptions{pdftex}
  \usepackage{graphicx}                % allow us to embed graphics files
  \DeclareGraphicsExtensions{.pdf,.png,.jpg,.jpeg} % for pdflatex we expect .pdf, .png, or .jpg files
\else%                                 % else we use pure latex
  \ExecuteOptions{dvips}
  \usepackage{graphicx}                % allow us to embed graphics files
  \DeclareGraphicsExtensions{.eps}     % for pure latex we expect eps files
\fi%

%% it is recomended to use ``\autoref{sec:bla}'' instead of ``Fig.~\ref{sec:bla}''
\graphicspath{{figures/}{pictures/}{images/}{./}} % where to search for the images

\usepackage{microtype}                 % use micro-typography (slightly more compact, better to read)
\PassOptionsToPackage{warn}{textcomp}  % to address font issues with \textrightarrow
\usepackage{textcomp}                  % use better special symbols
\usepackage{mathptmx}                  % use matching math font
\usepackage{times}                     % we use Times as the main font
\renewcommand*\ttdefault{txtt}         % a nicer typewriter font
\usepackage{cite}                      % needed to automatically sort the references
\usepackage{tabu}                      % only used for the table example
\usepackage{booktabs}                  % only used for the table example
\usepackage{amsmath}
\usepackage{subfig}
\usepackage{algpseudocode}
\usepackage{algorithm}
\usepackage{enumitem}
% \usepackage[belowskip=-10pt]{caption}


% \usepackage{geometry}
\usepackage{makecell}



% colors
\usepackage[dvipsnames]{xcolor}
\definecolor{ride_hailing_technology}{RGB}{242, 142, 44}
\definecolor{middle_east_threats}{RGB}{89, 161, 79}
\definecolor{controversies_and_challenges}{RGB}{78, 121, 167}
\definecolor{criminal_justice}{RGB}{153, 86, 51}
\definecolor{political_protests}{RGB}{130, 48, 207}
\definecolor{concerns_and_controversies}{RGB}{131, 230, 77}
\definecolor{presidential_election_and_controversies}{RGB}{48, 202, 129}
\definecolor{presidential_eligibility}{RGB}{240, 98, 130}
\definecolor{ben_carson}{RGB}{187, 187, 187}
\definecolor{vis_web_data}{RGB}{240, 129, 55}




% rgba(89, 161, 79, 0.498)

%% We encourage the use of mathptmx for consistent usage of times font
%% throughout the proceedings. However, if you encounter conflicts
%% with other math-related packages, you may want to disable it.


%% If you are submitting a paper to a conference for review with a double
%% blind reviewing process, please replace the value ``0'' below with your
%% OnlineID. Otherwise, you may safely leave it at ``0''.
\onlineid{1493}

%% declare the category of your paper, only shown in review mode
\vgtccategory{Research}

%% allow for this line if you want the electronic option to work properly
\vgtcinsertpkg

%% In preprint mode you may define your own headline. If not, the default IEEE copyright message will appear in preprint mode.
%\preprinttext{To appear in an IEEE VGTC sponsored conference.}

%% This adds a link to the version of the paper on IEEEXplore
%% Uncomment this line when you produce a preprint version of the article 
%% after the article receives a DOI for the paper from IEEE
%\ieeedoi{xx.xxxx/TVCG.201x.xxxxxxx}


%% Paper title.

\title{HyperMap: Analyzing large collections of documents with hypergraphs}


%% This is how authors are specified in the conference style

%% Author and Affiliation (single author).
%%\author{Roy G. Biv\thanks{e-mail: roy.g.biv@aol.com}}
%%\affiliation{\scriptsize Allied Widgets Research}

%% Author and Affiliation (multiple authors with single affiliations).
%%\author{Roy G. Biv\thanks{e-mail: roy.g.biv@aol.com} %
%%\and Ed Grimley\thanks{e-mail:ed.grimley@aol.com} %
%%\and Martha Stewart\thanks{e-mail:martha.stewart@marthastewart.com}}
%%\affiliation{\scriptsize Martha Stewart Enterprises \\ Microsoft Research}

%% Author and Affiliation (multiple authors with multiple affiliations)

\author{ Sam Yu-Te Lee\thanks{e-mail: ytlee@ucdavis.edu}\\ % \and Aryaman Bahukhandi\thanks{e-mail: abahukhandi@ucdavis.edu}\\ \and Kwan-Liu Ma\thanks{e-mail: klma@ucdavis.edu}\\ %
}
\affiliation{\scriptsize University of California, Davis}

%% A teaser figure can be included as follows
\teaser{
  \centering
  \includegraphics[height=8cm, keepaspectratio]{teaser}
  \caption{The HyperMap system. 
  (a) The peripheral area of Cluster View shows the mentioned characters of highlighted documents using Gilbert curves.
  (b) The center area of Cluster View shows the topic structure of the corpus using Gosper curves.
  (c) The Document View shows a list of selected documents.
  (d) The Chatbot View provides a chatbot interface to answer user questions with the option to insert selected documents in the prompt.}
\label{fig: sfc}
}

%% Abstract section.
\abstract{
Sensemaking on large collections of documents (corpus) is a challenging task that analysts often have to perform.
Previous works approach this problem either from a topic- or entity-based perspective, but they lack interpretability and trust due to poor model alignment.
In this paper, we propose HyperMap, a visual analytics approach that combines topic- and entity-based techniques seamlessly.
By leveraging the capability of Large Language Models (LLMs), we model the corpus as a hypergraph that matches the user's mental model when analyzing a corpus.
The hypergraph is then hierarchically clustered with an agglomerative clustering algorithm by combining semantic and connectivity similarity.
The system is designed to emphasize Model Alignment to foster interpretability and trust.
To demonstrate the generalizability and effectiveness of the HyperMap system, we present two case studies on two different datasets: a news article dataset and a visualization publication dataset.
We discuss limitations and future work of combining visualization and LLMs to enhance analysts' ability to analyze a corpus.
}


%% ACM Computing Classification System (CCS). 
%% See <http://www.acm.org/about/class> for details.
%% We recommend the 2012 system <http://www.acm.org/about/class/class/2012>
%% For the 2012 system use the ``\CCScatTwelve'' which command takes four arguments.
%% The 1998 system <http://www.acm.org/about/class/class/2012> is still possible
%% For the 1998 system use the ``\CCScat'' which command takes four arguments.
%% In both cases the last two arguments (1998) or last three (2012) can be empty.

% \CCScatlist{
%   \CCScatTwelve{Human-centered computing}{Visualization}{Visualization application domains}{Visual Analytics};
%   \CCScatTwelve{Information systems}{Information retrieval}{Retrieval tasks and goals}{Clustering and classification};
%   \CCScatTwelve{Human-centered computing}{Human computer interaction (HCI)}{Interaction paradigms}{Graphical user interfaces};
%   \CCScatTwelve{Applied computing}{Document management and text processing}{}{}
% }

%\CCScatlist{
  %\CCScat{H.5.2}{User Interfaces}{User Interfaces}{Graphical user interfaces (GUI)}{};
  %\CCScat{H.5.m}{Information Interfaces and Presentation}{Miscellaneous}{}{}
%}

%% Copyright space is enabled by default as required by guidelines.
%% It is disabled by the 'review' option or via the following command:
% \nocopyrightspace

%%%%%%%%%%%%%%%%%%%%%%%%%%%%%%%%%%%%%%%%%%%%%%%%%%%%%%%%%%%%%%%%
%%%%%%%%%%%%%%%%%%%%%% START OF THE PAPER %%%%%%%%%%%%%%%%%%%%%%
%%%%%%%%%%%%%%%%%%%%%%%%%%%%%%%%%%%%%%%%%%%%%%%%%%%%%%%%%%%%%%%%%

\begin{document}

%% The ``\maketitle'' command must be the first command after the
%% ``\begin{document}'' command. It prepares and prints the title block.

%% the only exception to this rule is the \firstsection command
% \firstsection{Introduction}

\maketitle
\section{Introduction}
Text data is ubiquitous.
From news articles and social media posts to scientific publications, the tremendous amount of text data that is produced poses not only opportunities but also a great challenge to anyone who needs to analyze them.
Visual analytics (VA) mitigates this challenge by combining mathematical models and visualizations to automate the sensemaking process and reduce the cognitive load.
Chuang et al.~\cite{chuang2012interpretation} proposed that \textit{Model alignment}, the alignment of analysis tasks, visual encodings and model decisions,
greatly affects users' interpretation and trust in visual analytic systems.
However, in text analysis, the available models often align poorly with analysis tasks.
For example, topic models are commonly used to model the topical structure of text documents.
Most topic models characterize \textit{topic} as a probabilistic distribution spanning a given vocabulary~\cite{vayansky2020review}.
This transformation from \textit{topics}, a high-level concept that the user seeks to understand, to a \textit{probabilistic distribution}, a low-level concept that mathematical models can operate on, prevents proper model alignment.
The misalignment between analysis tasks and models limits the usage of visual analytics systems for users who are not familiar with the underlying models.

Recent advances in large language models (LLM) present a promising solution to this problem.
LLMs have proven successful in various natural language processing (NLP) tasks, especially in question-answering tasks due to their strong capability to understand user intent.
Researchers in visualization have adopted LLMs to assist data transformation~\cite{wang2023dataformulator} or directly generate visualization~\cite{maddigan2023chat2vis}.
However, they all assumed a clean data format, where the data to be visualized is already in a table format. 
For unstructured text analysis though, this is rarely the case.
Topics~\cite{atzberger2023evaluatetopicmodel}, sentiments~\cite{beasley2021through}, concepts and entities~\cite{park2018conceptvector,cao2010facetatlas} are common analysis targets in text analysis, which require a data preparation stage to extract them from unstructured text.
Recently, Li et al.~\cite{li2023evaluateChatgpt} evaluated ChatGPT's capabilities on Information Extraction (IE) tasks comprehensively, and found that it excels under an OpenIE setting, where the model relies solely on user input to extract information from documents.
The capability of LLMs to extract information from documents according to user intent eliminates the need to carefully align the analysis tasks and models in VA systems,
because a specific model is no longer needed to prepare the data for the analysis task.
In the previous example, instead of relying on abstruse and unfathomable probabilistic models, LLMs can directly process the text data and summarize the topics of the documents.
A user can ask a LLM:\@ \textit{`What are the topics of these articles?'}, and the LLM would give a human-like response, such as \textit{`The articles are about \ldots'}.

However, using LLMs in the data preparation stage is not trivial. 
Problems like \textit{hallucination} and \textit{faithfulness} hinder the accuracy of the extracted information.
Token limits restrict the length of the input text, limiting the usage of LLMs on large collections of documents (corpus).
Prompts need to be carefully designed to reflect user intent.
Finally, the extracted information, the analysis task and the visualization need to be aligned to foster interpretation and trust.
In this work, we designed a VA system that models a corpus as a hypergraph, where the nodes are articles and salient entities (or concepts) extracted from the articles.
We showcase how LLMs are used flexibly to align the data, analysis task and visualization during our design process.
The hypergraph is then hierarchically clustered and visualized by extending space-filling curve layouts~\cite{muelder2008sfc}.
The system supports interactive exploration, reorganization and analysis of the documents.
To the best of our knowledge, no visual analytics system has adopted LLMs to assist the data preparation stage in text analysis.
Using the system, we demonstrate how proper model alignment can be achieved using LLMs.

The contributions of our work are as follows:
\begin{itemize}
    \item We introduce an LLM-based information extraction pipeline that is capable of extracting topics and salient entities from a given corpus in a way that fosters interpretation.
    \item We extend space-filling curve layouts to visualize clusters in large hypergraphs.
    \item We develop a novel VA system that allows users to effectively explore, reorganize and analyze a corpus.
\end{itemize}











\section{Related Works}
\subsection{Hypergraph visualization}
\cite{fischer2021hypergraphsurvey}
We decided to use node-link-based representation because it is more familiar to users~\cite{abdelaal2022network}.
\subsection{Summarizing large collections of text}
Topic models, entity-based summarization (VA approaches)
\subsection{Interpretability and Trust in text analysis}


\subsection{Design Rationale}~\label{sec: design_rationale}
HyperMap is designed for analysts to explore and reorganize a corpus for their analysis.
Our design rationale to foster interpretation is based on the design guidelines proposed by Chuang et al.~\cite{chuang2012interpretation}.
We reuse their definitions of \textit{Model Alignment}, \textit{Progressive Disclosure}, and \textit{Unit of Analysis} when describing our design rationale.
We first identify common analysis tasks from previous works.
Then, we derive our design considerations (DC) from the analysis tasks.
We take the DCs into account when making our model decisions and visualization design in~\autoref{sec: methodology}.
Finally, we explain how we achieve model alignment by applying the design considerations to our system.

\subsubsection{Analysis Tasks}
We derive our target analysis tasks from topic- and entity-based approaches.
Topic-based approaches aim to support document understanding by visualizing the topic structure of the documents.
Investigation of the topic structure seeks to answer the question: \textit{What topics are discussed in the corpus, and how are they related?}
Entity-based approaches support investigative analysis by visualizing entities and their relations.
Similarly, the investigation of the entities seeks to answer the question: \textit{What entities are involved in the corpus, and how are they related?}: 
We aim to support both tasks simultaneously as they are fundamental to subsequent tasks and intertwined in a real-world scenario.

\subsubsection{Design Considerations}
To support the aforementioned analysis tasks, we derive the following design considerations (DCs):
\begin{itemize}
  \item \textbf{DC1: Overview of topic structures and entity connections}
  Given a corpus, the topic structures and entity connections can be complex and cover a wide range of articles and entities. 
  The overview seeks to cover all the articles and entities by hiding the details.
  This sets the ground for the user to discover their targets of interest.
  \item \textbf{DC2: Progressive Disclosure}
  To facilitate investigation, it is important to support users to drill down from a high-level overview to intermediate abstractions.
  This includes disclosure of a specific topic's sub-structure, the containing articles, and the connections to entities.
  \item \textbf{DC3: Model Alignment}
  Our choice of model should align well with the analyst's mental model when conducting the analysis tasks.
  This means our model should directly operate on the units of analysis, which are topics (groups of similar articles) and entities.
  Then by properly visualizing the abstractions of the model, we are safe to produce a good model alignment.
  \item \textbf{DC4: Detailed analysis of the target of interest}
  The investigation of topic structures and entity connections often leads to a target of interest, which can be a topic or an entity.
  After such investigation, previous works usually only provide the user with a list of documents that are relevant to the target of interest.
  This is perhaps due to the lack of a unified way to analyze the target of interest under different contexts.
  The advance of LLMs presents a promising solution to this problem by transforming almost any analysis task into a question-answering task.
  We thus include this task to fill the gap in previous works.
\end{itemize}






\section{Methodology}
\subsection{Modeling}
\subsubsection{Hypergraph Construction}
A hypergraph is a generalization of a graph in which an edge can connect any number of nodes~\cite{fischer2021hypergraphsurvey}.
A hyperedge thus represents a multi-way relationship between nodes.
In this paper, we model two types of hypergraphs: article hypergraph and participant hypergraph, where articles and participants are the nodes, respectively.
\textit{Participants} are the core components that the article's content discuss~\cite{use_other_works_to_refine_definition}.
For example, in a news article, the participants can be named entities such as people, organizations, or locations.
In a research article, the participants can be the concepts or techniques used in the article.

Conducting analysis on article hypergraph and participant hypergraph correspond to topic-based and entity-based analysis, respectively.
Following the definition of a hypergraph node, a hyperedge can be used to represent two types of multi-way relationships:
(1) A hyperedge between \textit{articles} can be constructed if the articles all mention the same participant. 
In this case, the hyperedge represents the co-mention of a participant, i.e.\ a named entity or a concept;
(2) A hyperedge between \textit{participants} can be constructed if the participants are mentioned together in the same article.
In this case, the hyperedge represents a co-occurrence relationship between participants.
Once the two hypergraphs are constructed, they are hierarchically clustered separately.
Clusters in the article hypergraph represents topics that are discussed in the dataset.
Clusters in the participant hypergraph represents participants (entities or concepts) that frequently co-occurred in an article.
For better interpretability of the clustering result, we further assign \textit{tags} for each cluster, which is further explained in~\autoref{sec: tag_assignment}.

Although these two types of hyperedges are constructed differently, we utilize the \textit{dual} of a hypergraph to simplify the construction process.
The dual of a hypergraph is simply another hypergraph, where the hyperedges are now nodes and the nodes are now hyperedges. (Add formulas here to explain).
Therefore, we first model the articles as nodes and participants as hyperedges to construct the article hypergraph $H_A$.
The participant hypergraph $H_P$ can then be easily constructed by taking the dual of $H_A$.
This construction process also allows us to use the same clustering algorithm on both hypergraphs, which is further explained in~\autoref{sec: clustering}.

\subsubsection{Hierarchical Clustering}\label{sec: clustering}
Common clustering algorithms for graphs consider only graph connectivity.
However, for the best interpretability of the clustering result, the node embeddings must be also used in the clustering process.
The necessity of incorporating node embeddings is further explained in~\autoref{sec: tag_assignment}.
Therefore, this limits our choice of clustering algorithms to attributed node clustering algorithms.

Although there are existing approaches that can cluster attributed nodes on graphs such as EVA~\cite{citraro2020eva} and iLouvain~\cite{combe2015louvain}, they are not designed for hypergraphs.
In general, hypergraphs can be clustered in two different ways: 
(1) Directly operate on the hyperedges by generalizing the graph clustering algorithms.
For example, Kamiński et.al.~\cite{kaminski2021hgraphcommunity} generalizes the modularity metric for graphs to hypergraphs; 
(2) First transform the hypergraph into a graph and then apply normal graph clustering algorithms~\cite{kumar2020new}.
Although the first approach is more intuitive, it is less scalable and hard to incorporate node attributes.
Thus, we decided to design our clustering algorithm following the second approach.

Considering all the above, we implemented our hierarchical clustering algorithm by first transforming the hypergraph into a graph following the edge re-weighting process proposed by Kumar et.al.~\cite{kumar2020new},
then an agglomerative clustering algorithm~\cite{steinbach2000doccluster} is applied on the re-weighted graph.
In agglomerative clustering, the key is to define the similarity between nodes and similarity between clusters.
We can easily incorporate node attributes into the clustering process by defining the similarity between nodes and clusters as a combination of attribute similarity $S_s$ and connectivity similarity $S_c$.
Since we're dealing with texts, we refer to the attribute similarity between nodes as semantic similarity. 

The semantic similarity $S_s(i, j)$ is the cosine similarity of the embeddings of the two nodes, denoted as $v_i$.
For article nodes, the embeddings are generated using the article content.
For participant nodes, the embeddings are generated using a description note of the participant.
More details about the embeddings are explained in~\autoref{sec: embeddings}.
The connectivity similarity $S_c$ is the weighted Topological Overlap (wTO)~\cite{gysi2018wto},
which is a weighted generalization of the Overlap Coefficient~\cite{vijaymeena2016survey}, as shown in~\autoref{eq:connectivity_similarity}.
\begin{equation}\label{eq:connectivity_similarity}
    S_s(i, j) = \frac{v_i \cdot v_j}{||v_i|| \cdot ||v_j||}, \quad
    S_c(i, j) = \frac{\sum_{u=1}^N{w_{i,u}w_{u_j}} + w_{i,j}}{\min(k_i, k_j) + 1 - |w_{i,j}|}
\end{equation}
where $k_i = \sum_{j=1}^N |w_{i,j}|$ is the total weight of the edges connected to node $i$.
Finally, a weighting factor $\alpha$ is used to balance the two similarities, as shown in~\autoref{eq: similarity}.
\begin{equation}\label{eq: similarity}
    S = \alpha S_s + (1-\alpha) S_c
\end{equation}
For the similarity between clusters, we used centroid similarity, i.e.\ the similarity between two clusters is the similarity between the centroids of the two clusters.
The algorithm is presented in (TODO: add algorithm pseudocode here)



\subsection{Preprocessing}
The Methodology can work for any unstructured dataset
\subsubsection{Summarization}
Chatgpt for summarization
\subsubsection{Document Embedding}\label{sec: embeddings}
OpenAI's embedding API
\subsubsection{Participant Extraction}\label{sec: participant_extraction}
Chatgpt for major participant extraction and another model for entity linking

\subsubsection{Topic Assignment}\label{sec: tag_assignment}
Chatgpt to assign topics to each cluster


\section{Visualization}
\subsection{Space Filling Curves}
Introduce Gosper curve and generalized Hilbert curve, and how they are used for large graph layout

\subsection{SFC for HyperGraph}
Using the Gosper curve to layout the article graph

Concatenating four generalized Hilbert curve to layout the entity graph on the peripheral
\subsection{Spacing Strategy}

\subsection{Border Approximation}

\subsection{Edge Bundling}


\section{System Design}
\subsection{Cluster View}
Use SFC hypergraph to show topical structure of the dataset.

\subsubsection{Interactions}
\subparagraph{Click}
\subparagraph{Expansion}
\subparagraph{Filtering}
\subparagraph{Searching}

\subsection{Article View}

\subsection{Analysis View}
\section{Evaluation}\label{sec: evaluation}
\subsection{Case Study}
\subsubsection{AllTheNews Dataset}
\subsubsection{Vis Publication Dataset}
\input{sections/08_Limitations_and_Future_Work.tex}
% \vspace*{1cm}
\section{Conclusion}
In this paper, we propose HyperMap, a visual analytics system that assists users in exploring, reorganizing and analyzing a corpus.
Previous works have shown that topic- and entity-based analysis are essential to sensemaking on a corpus,
but existing text analysis tools do not provide a unified interface for both types of analysis.
We fill in this gap by combining state-of-the-art large language models and hypergraph analysis and visualization techniques.
We introduce an LLM-based pipeline to extract topics and characters from unstructured text documents.
We model the corpus as hypergraphs and apply an agglomerative clustering algorithm.
The clusters are visualized by building upon existing space-filling curve layouts, which exhibit a high level of visual scalability and aesthetics.
Moreover, we emphasize the importance of Model Alignment in the design of visual analytics systems.
The generalizability and effectiveness of HyperMap are demonstrated in two case studies, analyzing datasets from two different domains.
Our future research aims to utilize the capabilities of LLMs to understand user intent to support more advanced analysis tasks.







%% if specified like this the section will be committed in review mode
% \acknowledgments{
% The authors wish to thank A, B, and C. This work was supported in part by
% a grant from XYZ.}

%\bibliographystyle{abbrv}
\bibliographystyle{abbrv-doi}
%\bibliographystyle{abbrv-doi-narrow}
%\bibliographystyle{abbrv-doi-hyperref}
%\bibliographystyle{abbrv-doi-hyperref-narrow}

\bibliography{template}

\pagebreak
\appendix % You can use the `hideappendix` class option to skip everything after \appendix

% \newgeometry{left = 0.1cm,top=1cm,bottom=1cm}
\clearpage
\begin{center}
    \begin{tabular}{ | p{3.0cm} | p{3.0cm} | p{13cm} | }
        \hline
        \textbf{Task} & \textbf{Dataset} & \textbf{Prompt}\\
        \hline
        Summarization & All The News & \makecell[tl]{\{\\
        \quad\textbf{Role:} `system' \\
        \quad\textbf{Content:} `You are a summarization system that summarizes events happened between the\\
        \quad main characters of a news article. The user will provide you with a news article to summarize.\\
        \quad Try to summarize the article with no more than three sentences. \\
        \quad Reply starts with `The article discussed \dots \\
        \}
        } \\ 
        \hline
        Character Extraction & All The News & \makecell[tl]{\textit{Sub-task 1: Extract event }\\ 
        \{\\ 
        \quad \textbf{Role:} `system' \\
        \quad \textbf{Content:} `You are a state-of-the-art event extraction system. Your task is to extract only the \\
        \quad most important event from news articles. \\
        \quad Strictly extract only one event. This event should be the most important event in the article. \\
        \quad Also extract the trigger that indicates the occurrence of the event. \\
        \quad The events should be human-readable. \\
        \quad Reply in this format: [\textit{event} -- \textit{trigger}] '\\ 
        \}, \\
    \{\\ 
    \quad\textbf{Role:} `user', \textbf{Content:} `This is the news article:\{\textit{sentence}\}' \\ 
    \}\\
    \hline
    \textit{Sub-task2: Extract characters} \\ 
    \{\\
    \quad\textbf{Role:} `system' \\ 
    \quad\textbf{Content:} `You are a named entity recognition model. You will be given an article and the event \\
    \quad recognized in that article by the user.  \\
    \quad The format of the input defining event in article will be: [\textit{event - category}]; \\
    \quad Extract the main characters involved in that event. The number of main characters should be 2\\
    \quad or less strictly. Ignore any other characters other than the two main characters. \\
    \quad Be as concise as possible. Reply with the following format: [\textit{character 1}],[\textit{character 2}] \dots'\\
    \},\\
    \{\\
    \quad \textbf{Role:} `system', \textbf{Name:} `example\_user' \\
    \quad \textbf{Content:} `Article: The article discussed the extensive history of doping in Russia, dating back to the\\
    \quad 1983 Soviet Union's detailed instructions to inject top athletes with anabolic steroids in order to \\ 
    \quad ensure dominance at the Los Angeles Olympics. \dots \\
    %  Dr.Sergei Portugalov, a key figure in Russia's \\
    % \quad current doping scandal, was named as the mastermind behind the doping program. The revelations of  \\
    % \quad these schemes led to the banning of Russia's track and field  team from the Rio Games, the most \\
    % \quad severe doping penalty in Olympic history.'\\
    \quad Event: [Russia's doping scandal - sports scandal]\\
    \}, \\
    \{ \\
    \quad\textbf{Role:} `system', \textbf{Name:} `example\_system', \textbf{Content:} [Russia], [Dr.Sergei Portugalov]\\
    \}, \\
    \{\\
    \quad\textbf{Role:} `user', \textbf{Content: }`Article: \{\textit{sentence}\} Event: \{\textit{event}\}' \\
    \},
    } \\
    \hline
    \makecell[tl]{Character\\ Disambiguation}  & VisPub & \makecell[tl]{\{\\
    \quad\textbf{Role:} `system'\\
    \quad\textbf{Content:} `User will provide a list of keywords from research paper abstracts.'\\ 
    \quad The input will be in the format: keyword 1, keyword 2, \dots \\
    \quad Which phrase would best describe the list of keywords? \\
    \quad The phrase should be very specific and similar to the keywords and less than 5 words. \\
    \quad If the words are too similar to each other, simply assign one of the words as the topic. \\
    \quad Strictly assign one topic to a list of keywords. \\
    \quad Reply in the format: [\textit{research topic}]; \\
    \},\\
    \{ \\
    \quad\textbf{Role:} `system', \textbf{Name:} `example\_user', \textbf{Content:} `[storytelling, data storytelling]' \\
    \}, \\
    \{ \\
    \quad\textbf{Role:} `system', \textbf{Name:} `example\_system', \textbf{Content:} `[data storytelling]' \\
    \}, \\
    \{ \\
    \quad\textbf{Role:} `user', \textbf{Content:} `[{\textit{keywords}}]' \\   
    \} \\
    } \\
    \hline
    % \textbf{Role:} user &&&
    % \textbf{Content:} &&&"Main Characters:\textbackslash n$\{arguments\}$ \textbackslash n\textbackslash n\textbackslash n Article: $\{article\}$".format&&&(arguments=arguments, article=article)\\ 
    %  \hline

\end{tabular}
\end{center}

\clearpage
% \newgeometry{left = 0.1cm,top=1cm,bottom=1cm}
\begin{center}
  \begin{tabular}{ | p{3.0cm} | p{3.0cm} | p{13cm} | }
    \hline
    \textbf{Task} & \textbf{Dataset} & \textbf{Prompt}\\
    \hline
    \makecell[tl]{Topic Label Generation \\ for documents}  & All The News & \makecell[tl]{
    \textit{{Bottom Level:}} \\
    \{\\
    \quad\textbf{Role:} `system'\\
    \quad\textbf{Content:} `You are a news article summarization system.  \\
    \quad The user will provide you with a set of summarized news articles, your job is to further \\
    \quad summarize them into one noun phrase. \\
    \quad Use words that are already in the articles, and try to use as few words as possible.\\
    \},\\
    \{\\
    \quad\textbf{Role:} `system'\\
    \quad\textbf{Name:} `example\_user'\\ 
    \quad\textbf{Content:} `Article 1: The article discussed an engineering company that creates mobile research\\
    \quad robots for the military, which released a new video showcasing the `next generation' of its\\
    \quad humanoid Atlas robot. The footage showed the robot escaping and being tormented with a hock-\\
    \quad ey stick, but it quickly recovered. In December 2013, Google bought Boston, the company be-\\
    \quad hind the robots, \dots \\ 
    \quad Article 2: The article discussed the rise of robots and artificial intelligence (AI) in various\\
    \quad industries, including Amazon's commercial delivery to a customer in Cambridge, England and\\
    \quad the testing of driverless vehicles and drones. It also mentioned the potential job losses due to \\
    \quad automation, but argued that the panic is exaggerated and that new jobs will be created. \\
    \quad Article 3: \{\textit{summary of article 3 \dots}\}\\ 
    \quad Article 4: \{\textit{summary of article 4 \dots}\}' \\
    \},\\
    \{\\
    \quad\textbf{Role:} `system', \textbf{Name:} `example\_system'\\ 
    \quad\textbf{Content:} `Robotic Advancements and Concerns'\\
    % { "role": "user", "content": summaries_message}
    \}, \\
    \{\\
    \quad\textbf{Role:} `user', \textbf{Content:} `Article 1: \{\textit{article summary 1}\}, Article 2: \{\textit{article summary 2}\} \dots'\\ 
    \} \\
    \{ \\
    \quad\textbf{Role:} `system' \\
    \quad\textbf{Content:} `You are a news article categorization system. \\
    \quad The user will provide you with a list of sub-topics of news articles and a few examples from the\\
    \quad sub-topics. Your job is to further categorize the sub-topics into a single noun phrase that best\\
    \quad summarizes all the sub-topics. Try to reuse the words in the examples.\\
    \} \\ \\
    \hline
    \textit{Intermediate Level:}\\
    \{ \\
    \quad\textbf{Role:} `system' \quad\textbf{Name:} `example\_user'\\ 
    \quad\textbf{Content:} `Sub-Topics: Increasing Gun Violence in Chicago, Crime Rates and Policing Tactics,\\
    \quad Misconceptions about Crime in the United States, Global Events and Optimism\\
    \quad Article 1: The article discussed the major events of 2016, including the Orlando nightclub shoot-\\
    \quad ing, Bastille Day attack in Nice, French priest assassination, and cafe attack in Dhaka. \dots\\
    % It also mentioned\\
    % \quad the Brexit vote, police shootings of black men, reprisal shootings of officers, and the election of \\
    % \quad Donald Trump. However, Harvard professor Steven Pinker provided a different perspective, stating\\
    % \quad that overall, the world is still in a more peaceful period than at any other time in history, with\\
    % \quad improvements in poverty, child mortality, illiteracy, and global inequality. He also expressed\\
    % \quad conditional optimism for the future.\\
    \quad Article 2: The article discussed how serious crimes in the city increased last year while the \\
    \quad number of arrests decreased \dots \\
    % The murder rate jumped 4.5 percent and robberies rose by 2 percent. \\
    % \quad Marijuana arrests declined significantly, but gun busts went up. The NYPD Deputy Commissioner \\
    % \quad for Operations mentioned that arrests have been more targeted towards a `core population that is\\
    % \quad driving crime.' Manhattan and The Bronx saw an increase in overall crime.\\
    \}, \\
    \{\\
    \quad\textbf{Role:} `system' \quad\textbf{Name:} `example\_system'\\ 
    \quad\textbf{Content:} `Crimes in the United States'\\
    \},\\
    \\
    } \\
    \hline
\end{tabular}
\end{center}

\clearpage
% \newgeometry{left = 0.1cm,top=1cm,bottom=1cm}
\begin{center}
  \begin{tabular}{ | p{3.0cm} | p{3.0cm} | p{13cm} | }
    % \hline
    % \textbf{Task} & \textbf{Dataset} & \textbf{Prompt}\\
    \hline
    % \cline{2-3}
    & VisPub & \makecell[tl]{\textit{Bottom Level:}\\
    \{\\
    \quad\textbf{Role:} `system'\\
    \quad\textbf{Content:} `You are a visualization research paper summarization system. The user will provide \\
    \quad you with a set of abstracts of visualization research papers. They are manually categorized by \\
    \quad another person, so they are discussing the same topic.  Your job is to find out what that topic is.\\
    \quad Reply with less than five words. '\\
    \},\\
    \{\\
    \quad\textbf{Role:} `system' \quad\textbf{Name:} `example\_user'\\
    \quad\textbf{Content:} `Abstract 1: Many datasets such as scientific literature collections contain multiple \\
    \quad heterogeneous facets which derive implicit relations, as well as explicit relational references \\
    \quad between data items \dots \\ 
    \quad Abstract 2: We present PivotPaths, an interactive visualization for exploring faceted information\\
    \quad resources \dots \\
    \},\\
    \{\\
    \quad\textbf{Role:} `system' \quad\textbf{Name:} `example\_system'\\
    \quad\textbf{Content:}`Faceted browsing visualization'\\
    \},\\
    \{\\
    \quad\textbf{Role:} `user'\\
    \quad\textbf{Content:} `Abstract 1: \{\textit{abstract 1}\}, Abstract 2: \{\textit{abstract 2}\} \dots'\\
    \}\\
    \\
    \hline
    \textit{Intermediate Level:}\\
    \{\\
    \quad\textbf{Role:} `system'\\
    \quad\textbf{Content:} `You are a visualization research paper summarization system. \\
    \quad You generate topics for a set of visualization research papers.\\
    \quad The user will provide you with a list of sub-topics and a few example abstracts from the sub-\\
    \quad topics. Your job is to further categorize the sub-topics into a single noun phrase that best \\
    \quad summarizes all the sub-topics. Try to reuse the words in the sub-topics.\\
    \quad Reply with a single noun phrase without any line breaks. Be as concise as possible.'\\
    \},\\
    \{\\
    \quad\textbf{Role:} `system' \quad\textbf{Name:} `example\_user'\\
    \quad\textbf{Content:} `Sub-Topics: Underwater 3D scene reconstruction from acoustic imaging sonar data, \\
    \quad Visualization of Sound Propagation in Room Acoustics, Underwater seabed visualization;\\
    \quad Abstract 1: The development of a high speed multi-frequency continuous scan sonar at Sonar\\
    \quad Research Development Ltd has resulted in the acquisition of extremely accurate, high resolution\\
    \quad bathymetric data. \dots\\
    \quad Abstract 2:  \dots '\\
    \},\\
    \{\\
    \quad\textbf{Role:} `system' \quad\textbf{Name:} `example\_system'\\
    \quad\textbf{Content:} `Visualization of Underwater Acoustic Scenes and Sound Propagation'\\
    \},\\
    \{\\
    \quad\textbf{Role:} `user'\\
    \quad\textbf{Content:} `Sub-Topics: \{\textit{sub topics}\}, Abstract 1: \{\textit{abstract 1}\}, Abstract 2: \{\textit{abstract 2}\}, \dots'\\
    \}\\
    }\\
    \hline
\end{tabular}
\end{center}

\clearpage
% \newgeometry{left = 0.1cm,top=1cm,bottom=1cm}
\begin{center}
  \begin{tabular}{ | p{3.0cm} | p{3.0cm} | p{13cm} | }
    \hline
    \textbf{Task} & \textbf{Dataset} & \textbf{Prompt}\\
    \hline
    Topic Label Generation for characters & All The News & \makecell[tl]{
    \{\\
    \quad\textbf{Role:} `system'\\
    \quad\textbf{Content:} `You are an entity summarization system.\\
    \quad The user will provide you with a list of entities, they can be people, places, or things.\\
    \quad The user wants to get a gist of what entities are in the list.\\
    \quad First, split the entities into different categories.\\
    \quad Then, assign each category a human-readable name.\\
    \quad If entities in a category are all related to a specific entity, use that entity as the category.\\
    \quad Limit the number of categories to be less than 5 by keeping only the important categories. \\
    \quad Reply with the following format: Category 1, Category 2, Category 3, \dots \\
    \quad Do not reply more than 5 categories.'\\
    \},\\
    \{\\
    \quad\textbf{Role:} `system', \textbf{Name:} `example\_user\_1'\\ 
    \quad\textbf{Content:} `Entities: Refugee, Refugee camp, Internally displaced person, Asylum seeker'\\
    \}, \\
    \{\\
    \quad\textbf{Role:} `system', \textbf{Name:} `example\_system\_1', \textbf{Content:} `Refugee'\\
    \},\\
    \{\\
    \quad\textbf{Role:} `system', \textbf{Name:} `example\_user\_2'\\ 
    \quad\textbf{Content:} `Entities: JetBlues, American Airlines, Delta Airlines, Southwest Airlines, United\\
    \quad Airlines, Alaksa Airlines'\\
    \}, \\
    \{\\
    \quad\textbf{Role:} `system', \textbf{Name:} `example\_system\_2', \textbf{Content:} `US airline companies'\\
    \},\\
    \{\\
    \quad\textbf{Role:} `user', \textbf{Content:} `Entities: \{\textit{Entity List}\}'\\
    \},\\
    } \\
    \cline{2-3}
     & VisPub & \makecell[tl]{
    \{\\
    \quad\textbf{Role:} `system'\\
    \quad\textbf{Content:} `You are a visualization research paper keyword summarization system.\\
    \quad The user will provide you with a list of keywords, they are terminologies in visualization research\\
    \quad papers. The user wants to get a gist of what keywords are in the list, but the list is too long.\\
    \quad Pick out only a few keywords that best represents the list. Avoid picking out overlapping keywords. \\
    \quad Limit the number of picked keywords to be less than 5 by keeping only the important ones. \\
    \quad Reply with the following format: Keyword\_1, Keyword\_2, Keyword\_3, \dots \\
    \},\\
    \{\\
    \quad\textbf{Role: } `user', \textbf{Content:} `Keywords: \{\textit{keyword list}\}' \\
    \}\\
    }\\
    \hline
\end{tabular}
\end{center}
\clearpage
\section{Chat Responses for usage scenario 2}
\quad \textit{Q: } Given the selected documents, what should I consider when designing text visualization?

\textit{A: }
When designing text visualization, here are some key considerations based on the selected articles:\\
When designing text visualization, you should consider the following aspects based on the selected articles: \\
1. Amount of text: Determine the appropriate amount of text to include in the visualization. The articles suggest that adding text annotations to visualizations can enhance understanding and should not be penalized. Participants preferred charts with more textual annotations over charts with fewer annotations or text alone. \\
2. Semantic content: Consider the type of information conveyed by the text. Text that describes statistical or relational components of a chart leads to more takeaways referring to statistics or relational comparisons. \\
3. Placement of text: Text should be placed strategically within the visualization. Different kinds of information are best placed in the title or closer to the data, depending on the semantic level. \\
4. Integration of text and charts: Aim for a tight coupling between text and charts to support the synthesis of information. Investigate ways to establish interactive references between text and charts, allowing users to construct connections effortlessly. \\
5. Real-time analysis: If dealing with real-time data, consider the scalability and adaptiveness of the visualization. Develop interactive systems that enable visual analysis of streaming social media data or dynamically cluster posts related to relevant topics. \\
6. Support for specific tasks: Tailor the visualization to specific tasks such as clinical chart review, argument structure analysis, or anomaly detection in social media. Incorporate insights from domain experts and conduct iterative design processes to ensure usability. \\
7. Enhancement of comprehension: Explore techniques that facilitate enhanced comprehension of complex documents by coupling text content with data visualizations. Generate on-demand visualizations based on the reader's focus within the document and evaluate their effectiveness against traditional reading experiences. \\
8. Customizability and integration with existing workflows: Provide tools that are customizable to the needs of users, such as healthcare providers and physicians, while preserving and extensively linking to the original text. Avoid obscuring or substituting the original clinical text that contains qualitative and rhetorical flavors. \\
9. Facilitating analysis of complex textual differences: Develop visual analytics environments that support the alignment of complex textual differences, prevailing in medieval text traditions or other literature domains. Provide interactive visualizations for investigating repetition, variance, and other textual hierarchies. \\
10. Evaluation and validation: Perform user studies, expert reviews, and evaluations to validate the effectiveness and utility of the visualization techniques. Collect feedback from users and iterate designs based on their needs and preferences. Consider these guidelines when designing text visualizations to ensure effective communication and comprehension of information.\\

\textit{Q: } Given the selected documents, what is considered a balanced ratio of text and charts?

\textit{A: }
Based on the selected article, it is suggested that a balanced ratio of text and charts can vary depending on the context and purpose of the visualization. 
The article mentions that heavily annotated charts were not penalized and participants preferred charts with more textual annotations. 
However, it also emphasizes the importance of considering the qualitative properties and placement of the text within the chart. 
Overall, it is recommended to use a combination of text and charts that provides additional context, guidance, and relevant information to the readers without overwhelming or distracting them from the visualizations.

\section{Agglomerative Clustering}
The algorithm takes a hypergraph $H=(V, E)$ and the embeddings of each node $Vec$ as input, and outputs a sequence of partitions $P=P_1, P_2, \dots P_k$.
Each partition corresponds to a level in the hierarchy.
\begin{algorithm}\label{alg: clustering}
    \caption{Agglomerative Clustering }\label{alg:cap}
    \hspace*{\algorithmicindent} \textit{Input}: $H=(V, E)$, $Vec=\{vec_i |\, i \in V\}$ \\
    \hspace*{\algorithmicindent} \textit{Output}: Sequence of partitions $\mathbf{P}=P_1, P_2, \dots P_k$ \\
    \begin{algorithmic}[1]
    \While{$|V| > 1$}
        \State$P_k = init\_partition(V)$ \Comment{Initialize each node as a cluster}
        \State$\mathbf{S_s} = cosine\_similarity(V\times V, Vec)$ 
        \State$\mathbf{S_c} = wTO(V\times V, E)$ \
        \For{$i \in V$}
            \State$j = most\_similar\_node(i, S_s, S_c)$ 
            \State$P_k = merge\_clusters(i, j)$ \Comment{Merge the two clusters}
        \EndFor
        \State$H^\prime=(V^\prime, E^\prime) = construct\_hypergraph(P_k)$ \Comment{clusters are the new nodes}
        \State$Vec^\prime = centroid\_similarity(V^\prime)$
        \State$V=V^\prime, E=E^\prime, Vec=Vec^\prime$ \Comment{Update for next iteration}
    \EndWhile
    \end{algorithmic}
\end{algorithm}


% Refer to \cref{sec:appendices_inst} for instructions regarding appendices.

\end{document}
