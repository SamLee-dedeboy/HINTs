\section{Related Works}
\subsection{Data preparation for large collections of text}
\subparagraph{Topic-based approaches} 
Topic-based approaches employ certain variations of topic models to organize the documents in a meaningful way.
Each topic is often presented as a `bag of words', which can be in the form of a sequence of words~\cite{yang2017vistopic, cho2016vairoma, dou2013hierarchicaltopics,yan2019clfsemi,han2022hisva, alexander2014serendip, lee2012ivisclustering} or word clouds~\cite{oelke2014topiccoin, cho2016vairoma}.
The modeling result provides an overview of the dataset for subsequent analysis tasks. 
Despite their variations in model choices, the use of topic models as an overview, as well as its `bag-of-word' visualization, is reported by Lee et al.~\cite{lee2017humantopicmodel} to be problematic in a comprehensive user study, especially for non-expert users.
First, the discovered topics are not always semantically meaningful in human evaluation~\cite{chang2009reading}.
Second, the `bag of words' representation causes misinterpretation of the modeling results.
Users sometimes overlooked important words, assumed association between adjacent words, or misinterpreted the meaning of a word due to lack of context.
Chuang et al.~\cite{chuang2012interpretation} concluded that these problems arise from a misalignment between the analysis task, visual encoding and model.
Most systems employ topic models due to a need to provide an overview of the documents.
When topic models are employed to conduct such an analysis task, a transformation from semantically meaningful \textit{topics} to probabilistic distribution over words is done.
Most systems design the visualization in a way that this transformation is hidden for non-expert users because it is unnecessarily complex and unrelated to the analysis task.
However, the sensemaking process becomes challenging without a basic understanding of the model because the `bag-of-words' representation is too far away from the user's mental picture of a topic.
This misalignment limits the usage of topic models for non-expert users and makes the system prone to produce false positives.

\subparagraph{Entity-based approaches}
A line of work that makes successful model alignments is the entity-based approach.
`Entities' usually include named entities (people, organizations, locations), or meaningful concepts known to an existing knowledge base.
The earliest of such approaches is Jigsaw~\cite{Stasko2007jigasw}, where entities are linked if they appear in the same document.
FacetAtlas~\cite{cao2010facetatlas} generalizes the idea of entity to `facets' which can be entities or any keywords or user's interest.
ConceptVector~\cite{park2018conceptvector} uses `concept' to represent a similar idea. 
Although the system is also built for document analysis, it focuses on enabling users to build these concepts in a semi-automatic manner rather than extracting them fully automatically.
Generally, entity-based approaches exhibit better model alignments than topic-based approaches~\cite{chuang2012interpretation}, 
but the polysemy of natural language makes them prone to produce false positives~\cite{park2018conceptvector}.

\subparagraph{Embedding-based approaches}
Finally, an important line of work organizes documents by directly modeling their semantic similarity~\cite{steinbach2000doccluster}.
Documents are first projected into a high-dimensional vector space where similarity can be measured, and then a dimensionality reduction technique (e.g.t-SNE) is used to project the dataset onto a two-dimensional space for visualization.
Proximity in the reduced dimension represents similarity between documents.
Earlier works construct a sparse vector using term-frequency based scores such as \textit{TF-IDF} or BM25~\cite{choo2013utopian,sherkat2018interactive}.
More recently, the success of pre-trained language models like BERT~\cite{devlin2018bert} popularizes the idea of embedding documents in a dense vector space~\cite{narechania2022vitality,tu2023sdrquerier,qiu2022docflow}.
The embedding can then be used for document retrieval~\cite{karpukhin-etal-2020-dense, izacard2022unsupervised} or visualization.
Embedding-based approaches also exhibit healthy model alignment, as the vector space directly models the analysis task (finding similar documents). 
However, the result often lacks explainability and prevents users from trusting the result.  (TODO: cite here to support the claim?)

\subsection{LLMs for Information Extraction}
Information Extraction aims to identify structured information of interest from unstructured text data.
Some of its subtasks include Named Entity Recognition (NER), Relation Extraction (RE) and Event Extraction (EE)~\cite{nasar2021named, xiang2019surveyee}.
Although LLMs have proven successful in many NLP tasks, their application to IE is non-trivial.
First, the \textit{faithfulness} of LLMs needs to be carefully evaluated.
Faithfulness refers to the ability of a model to adhere to the provided information and not use parametric knowledge learned during training to answer user questions~\cite{zhou2023contextfaithful}.
When conducting information extraction, it is necessary to ensure that the extracted information is actually from the provided text and not from the model's parametric knowledge.
Second, the \textit{hallucination} problem of LLMs, where LLMs provide answers factually contradicting to input text (intrinsic) or even factually false (extrinsic). 
In the context of IE, we mainly focus on the intrinsic hallucination problem.
A recent evaluation conducted by Bang et al.~\cite{bang2023multitask} found that ChatGPT rarely exhibits intrinsic hallucinations, including the abstractive summarization task from which neural models usually suffer.

More specifically, Li et al.~\cite{li2023evaluateChatgpt} comprehensively evaluated the capabilities of ChatGPT for common IE tasks.
They found that ChatGPT excels under the Open-IE setting, where the model relies solely on user input to extract information from documents, but performs poorly under the Standard-IE setting, where ChatGPT is instructed to choose a correct label.
Their findings agree with Zhang et al.~\cite{zhang2023extractive} where ChatGPT is reported to perform poorly on extractive summarization.
A common reason for the poor performance of ChatGPT in these tasks is that they are essentially supervised learning tasks, and ChatGPT is not trained to perform them.
To make the best use of ChatGPT (or more generally, LLMs) for IE tasks, we need to carefully design the extraction tasks as question-answering tasks instead of supervised learning tasks.

% \cite{zamfirescu2023prompt, sultanum2023datatales} prompting is hard for users
\subsection{Hypergraph Visualization}
A hypergraph is a generalization of a graph in which an edge can connect any number of nodes.
Fischer et al.~\cite{fischer2021hypergraphsurvey} divides existing static hypergraph visualization approaches into node-link-based and matrix-based approaches.
Although matrix-based approaches are known for their visual scalability, they are not suitable for our system because of their lack of support for visualizing hierarchical clusters and user interactions.
We thus focus on node-link-based approaches in this section.

Node-link-based approaches are the most common and intuitive way of visualizing hypergraphs, as they directly extend existing node-link visualization on graphs to hypergraphs.
One line of work treats hyperedges as set membership relations between nodes, thus visualization of set memberships can be directly employed.
Many works visualize set memberships by extending the Euler diagram~\cite{collins2009bubble, riche2010euler, simonetoo2016euler} using colored contours to indicate different sets, but the visual scalability is limited to small hypergraphs. 
KelpDiagram~\cite{dinkla2012kelp} and KepFusion~\cite{meulemans2013kelpfusion} use a kelp metaphor and improve visual scalability by optimizing link sizes.
However, they do not consider optimizing point positions and thus are most suitable for geospatial applications.
MetroSets~\cite{jacobsen2020metrosets} uses a metro map metaphor in which hyperedges are represented as metro lines and nodes are represented as stations.
By optimizing the point positions to mimic a metro map, they achieved high visual scalability and pleasing aesthetics.
Kerren et al.~\cite{kerren2013novel} uses a radial layout where points are arranged in a circle arcs around the circle represent hyperedges.
However, the quadratic runtime limits the system to no more than 200 vertices and 20 hyperedges.

A second line of work represents hyperedge as a polygon so the vertices of the hyperedges are also the vertices of the polygon~\cite{qu2017interactive, qu2021automatic, oliver2023scalable}.
Colors of the polygon are used to indicate the cardinality of the hyperedge.
However, these approaches have a serious overlapping issue on non-planer hypergraphs, thus limiting their scalability.

Finally, some node-link-based approaches visualize hypergraphs by converting them into graphs.
The most naive way of converting is through clique expansion, where each hyperedge is expanded into a clique.
This conversion is known to introduce dense edge crossings and ambiguity, as one can not tell whether two nodes are connected by a hyperedge or by a path of hyperedges~\cite{ouvrard2017hypergraph}. 
Instead, Ouvrard et al.~\cite{ouvrard2017hypergraph} proposed an extra-node representation, where extra nodes are added to represent hyperedges, essentially transforming the hypergraph into a bipartite graph.
We find the extra-node representation most scalable and intuitive, thus our design direction follows this approach.