\section{HyperMap System Design}
Below, we describe the HyperMap frontend system, including the views, visualizations and interactions, and how they assist exploration and reorganization of a corpus.
The HyperMap system consists of three main views: Cluster View, Article View and Analysis View.
The Cluster View visualizes the corpus as two hypergraphs using the SFC layout described in~\autoref{sec: visualization}.
The Document View shows the articles along with necessary statistics when the user makes a selection in the Cluster View.
The Analysis View integrates an LLM-based chatbot to assist the user in analyzing the selected articles. 
\textbf{TODO: add labels to each view in figure 1}
\subsection{Cluster View}
Cluster View is the main view of the HyperMap system.
It visualizes the corpus as two hypergraphs using the SFC layout described in~\autoref{sec: visualization}.
Using the Cluster View, users can explore the topic structure and entity connections simultaneously (\textbf(DC1)).
Below, we mainly discuss user interactions and the coordination between the Cluster View and other views.
\subsubsection{Interactions}
\subparagraph{Hover and click}
By default, the cluster labels are hidden to reduce clutter.
Hovering over the clusters will trigger a highlight effect and show the cluster label, indicating to users that it is an interactive object.
Users can select a cluster by clicking on the cluster label.
This will also trigger the Article View to show articles within the cluster and the connected entities (\textbf{DC1}).
Additionally, clicking on the cluster will temporarily expand the cluster to expose its sub-structure (\textbf{DC2}), as shown in~\autoref{fig: sub-cluster}.
The sub-structure is colored in different colors while maintaining the original cluster's shape.
The labels of each sub-cluster are also shown radially.
When the user clicks on any cluster or cluster label, the connected characters are highlighted and others fade out.
Under such cases, hovering over the character clusters will show not only the cluster label but also the highlighted characters in a list, as shown in~\autoref{fig: character-cluster-label}.
Users can click on the connected characters to filter the articles in the Article View.

\subparagraph{Expansion}
The expansion operation breaks a cluster into smaller clusters.
This operation is necessary because the agglomerative clustering result is not always semantically optimal.
Some clusters may be too vague, while others may be too specific.
By breaking down a cluster, users can also investigate a level deeper into the topic structure (\textbf{DC2}).
We use (Cmd + Click) or (Ctrl + Click) to expand a cluster.
The sub-cluster will redistribute the spacing of their parent cluster proportionally to their size.
This ensures that other clusters are not moved in the layout.
\textbf{TODO: figure}

\subparagraph{Filtering}
At any point in the exploration when users find that they have found the target of interest,
they can click the filter button to remove irrelevant documents and characters from the view.
The filtering functionality effectively creates a sub-hypergraph based on node selection, which supports any interactions that are available in the original hypergraph.
Note that we only support filtering based on document node selection, and character nodes that are not connected to the selected document nodes are filtered accordingly.
Although filtering based on character nodes is technically possible, we do not find the operation intuitive.
We decided to remove this feature to prevent users from losing themselves in the exploration process.

\subparagraph{Searching}
HyperMap supports searching by document embeddings.
Users can create any query in natural language, and the system will return the most relevant documents.
The search functionality is implemented by ranking the documents based on their cosine similarity to the query.
The user query is first embedded into the same vector space as the documents.
Then the cosine similarity between the query and each document is calculated.
The server returns the ranked documents to the front end, and the user can control the number of documents to be highlighted by a relevancy threshold. 
The highlighted documents and connected characters are visually distinguished in the Cluster View.
\textbf{TODO: figure}

\subsection{Document View}
The Document View displays a list of articles in a cluster.
Users can click any cluster or sub-cluster label to inspect the articles within.
The document view is colored according to the cluster color in the Cluster View.
Each document is an interactive card, with the title, summary, ID and relevancy score (if available) shown.
If the user is under search mode, documents above the relevancy threshold are also highlighted.
Users can click any document card to add or remove it in the Analysis View.

\subsection{Analysis View}
The Analysis View is a chatbot that assists users in analyzing the selected documents (\textbf{DC4}).
Users can use the summary or the full content of the selected document to ask any questions in natural language.
For example, ...

