\section{HyperMap System Design}
Below, we describe the HyperMap frontend system, including the views, visualizations and interactions, and how they assist exploration and reorganization of a corpus.
The HyperMap system consists of three main views: Cluster View, Article View and Analysis View.
The Cluster View visualizes the corpus as two hypergraphs using the SFC layout described in~\autoref{sec: visualization}.
The Document View shows the articles along with necessary statistics when the user makes a selection in the Cluster View.
The Analysis View integrates an LLM-based chatbot to assist the user in analyzing the selected articles. 
\subsection{Cluster View}
Cluster View is the main view of the HyperMap system (\autoref{fig: sfc}-a and -b).
It visualizes the corpus as two hypergraphs using the SFC layout described in~\autoref{sec: visualization}.
Using the Cluster View, users can explore the topic structure and character connections simultaneously (\textbf{DC1}).
Below, we mainly discuss user interactions and the coordination with other views.
\vspace*{-0.1cm}
\subsubsection{Interactions}
\subparagraph{Hover and click}
By default, the cluster labels are hidden to reduce clutter.
Hovering over the clusters will trigger a highlight effect and show the cluster label, indicating to users that it is an interactive object.
Users can select a cluster by clicking on the cluster label.
This will trigger the Article View to show articles in the cluster and the mentioned characters (\textbf{DC1}).
Additionally, clicking on the cluster will temporarily expand the cluster to expose its sub-structure (\textbf{DC2}), as shown in~\autoref{fig: case_1}-b.
The sub-structure is colored in different colors while maintaining the original cluster's shape.
The labels of each sub-cluster are also shown radially.
When the user clicks on any cluster or cluster label, the mentioned characters are highlighted and others fade out.
Under such cases, hovering over the character clusters will show not only the cluster label but also the highlighted characters in a list, as shown in~\autoref{fig: sfc}-a.
Users can also click on the mentioned characters to filter the articles in the Article View.

\vspace*{-0.3cm}
\subparagraph{Expansion}
The expansion operation breaks a cluster into smaller clusters.
This operation is necessary because the agglomerative clustering result is not always semantically optimal, and some clusters may be too vague for the user.
By breaking down a cluster, users can investigate a level deeper into the topic structure (\textbf{DC2}).
We use (Cmd + Click) or (Ctrl + Click) to expand a cluster.
When clicked, a smooth animation triggers showing how the parent cluster is broken down into smaller pieces.
The sub-clusters will redistribute the spacing of their parent cluster proportionally to their size.
This ensures that only local changes in the layout need to be made when expanding a cluster.

\vspace*{-0.3cm}
\subparagraph{Filtering}
At any point in the exploration when users find that they have found the target of interest,
they can click the filter button to remove irrelevant documents and characters from the view.
The filtering functionality effectively creates a sub-hypergraph based on node selection, supporting any interactions that are available in the original hypergraph.
Note that we only support filtering based on document node selection, and character nodes that are not mentioned in the selected documents are filtered accordingly.
Although filtering based on character nodes is technically possible, we do not find the operation intuitive.
We decided to remove this feature to prevent users from losing themselves in the exploration process.

\vspace*{-0.3cm}
\subparagraph{Searching}
HyperMap supports searching by document embeddings.
Users can create any query in natural language, and the system will return the most relevant documents (above \autoref{fig: sfc}-c).
The search functionality is implemented by ranking the documents based on their relevancy to the query using cosine similarity.
The user query is first embedded into the same vector space as the documents, then the relevancy score between the query and each document is calculated.
The server returns the documents sorted by relevancy score to the front end, and the user can control the number of documents to be highlighted by a relevancy threshold. 
The highlighted documents and mentioned characters are visually distinguished in the Cluster View (\autoref{fig: case_1}-b).

\subsection{Document View}
The Document View displays a list of selected articles (\autoref{fig: sfc}-c).
It can be a list of documents in a cluster or a list of documents that are returned by the search functionality.
Each document is an interactive card, with the title, summary, ID and relevancy score (if available) shown.
The main characters of the document are highlighted in yellow in the summary to assist quick browsing.
If the user is under search mode, documents are sorted according to relevancy and those above the relevancy threshold are highlighted.
Users can click any document card to add or remove it as `query documents'.
The query documents can then be used to ask questions in the Analysis View.

\subsection{Analysis View}
The Analysis View (\autoref{fig: sfc}-d) is a chatbot that assists users in analyzing the selected documents (\textbf{DC4}).
We designed the interface by mimicking a typical chatbot interface like ChatGPT.
We use the OpenAI `gpt-3.5-turbo-16k-0613' to generate responses.
All chat history is included in the prompt unless the user clears the chat history.
In addition to common chatbot functionalities, we support inserting selected documents into the user prompt.
Users can click on any document card in the Document View to add it to the prompt.
The chatbot can then give answers to user questions based on the selected documents.
They can also select if they want to insert the summary or the full content of the document.

