\section{Methodology}
\subsection{Modeling}
\subsubsection{Hypergraph Construction}
A hypergraph is a generalization of a graph in which an edge can connect more than two nodes~\cite{hypergraph_definition}.
A hyperedge thus represents a multi-way relationship between nodes.
In this paper, we model two types of hypergraphs: article hypergraph and participant hypergraph, where articles and participants are the nodes, respectively.
\textit{Participants} are the core components that the article's content discuss~\cite{use_other_works_to_refine_definition}.
For example, in a news article, the participants can be named entities such as people, organizations, or locations.
In a research article, the participants can be the concepts or techniques used in the article.

Following the definition of a hypergraph node, a hyperedge can be used to represent two types of multi-way relationships:
(1) A hyperedge between \textit{articles} can be constructed if the articles all mention the same participant. 
In this case, the hyperedge represents the co-mention of a participant, i.e.\ a named entity or a concept;
(2) A hyperedge between \textit{participants} can be constructed if the participants are mentioned together in the same article.
In this case, the hyperedge represents a co-occurrence relationship between participants.

Although these two types of hyperedges are constructed differently, we utilize the \textit{dual} of a hypergraph to simplify the construction process.
The dual of a hypergraph is simply another hypergraph, where the hyperedges are now nodes and the nodes are now hyperedges. (Add formulas here to explain).
Therefore, we first model the articles as nodes and participants as hyperedges to construct the article hypergraph $H_A$.
Then we apply a hierarchical clustering algorithm on $H_A$.
The detail of the clustering algorithm is explained in~\autoref{sec: clustering}.
The result of the clustering algorithm represents topics that are discussed in the articles.
Then, we apply the same clustering algorithm on the dual of the hypergraph, which is the participant hypergraph $H_P$.
The result represents groups of participants that are frequently mentioned together in the articles.

\subsubsection{Hierarchical Clustering}\label{sec: clustering}
Common clustering algorithms for graphs consider only graph connectivity.
However, for the best interpretability of the clustering result, the node embeddings must be also used in the clustering process.
The necessity of incorporating node embeddings is further explained in~\autoref{sec: topic_assignment}.
Therefore, this limits our choice of clustering algorithms to attributed node clustering algorithms.

Although there are existing approaches that can cluster attributed nodes on graphs such as EVA~\cite{citraro2020eva} and iLouvain~\cite{combe2015louvain}, they are not designed for hypergraphs.
In general, hypergraphs can be clustered in two different ways: 
(1) Directly operate on the hyperedges by generalizing the graph clustering algorithms.
For example, Kamiński et.al.~\cite{kaminski2021hgraphcommunity} generalizes the modularity metric for graphs to hypergraphs; 
(2) First transform the hypergraph into a graph and then apply normal graph clustering algorithms~\cite{kumar2020new}.
Although the first approach is more intuitive, it is less scalable and hard to incorporate node attributes.
Thus, we decided to design our clustering algorithm following the second approach.

Considering all the above, we implemented our hierarchical clustering algorithm by first transforming the hypergraph into a graph following the edge re-weighting process proposed by Kumar et.al.~\cite{kumar2020new},
then an agglomerative clustering algorithm~\cite{steinbach2000doccluster} is applied on the re-weighted graph.
In agglomerative clustering, the key is to define the similarity between nodes and similarity between clusters.
We can easily incorporate node attributes into the clustering process by defining the similarity between nodes and clusters as a combination of attribute similarity $S_s$ and connectivity similarity $S_c$.
Since we're dealing with texts, we refer to the attribute similarity between nodes as semantic similarity. 

The semantic similarity $S_s(i, j)$ is the cosine similarity of the embeddings of the two nodes, denoted as $v_i$.
For article nodes, the embeddings are generated using the article content.
For participant nodes, the embeddings are generated using a description note of the participant.
More details about the embeddings are explained in~\autoref{sec: embeddings}.
The connectivity similarity $S_c$ is the Jaccard similarity, as shown in~\autoref{eq:connectivity_similarity}.
\begin{equation}\label{eq:connectivity_similarity}
    S_s(i, j) = \frac{v_i \cdot v_j}{||v_i|| \cdot ||v_j||}, \quad
    S_c(i, j) = \frac{|N_i \cap N_j|}{|N_i \cup N_j|}
\end{equation}
A weighting factor $\alpha$ is used to balance the two similarities, as shown in~\autoref{eq: similarity}.
\begin{equation}\label{eq: similarity}
    S = \alpha S_s + (1-\alpha) S_c
\end{equation}
For the similarity between clusters, we used centroid similarity, i.e.\ the similarity between two clusters is the similarity between the centroids of the two clusters.
The algorithm is presented in (TODO: add algorithm pseudocode here)



\subsection{Preprocessing}
The Methodology can work for any unstructured dataset
\subsubsection{Summarization}
Chatgpt for summarization
\subsubsection{Document Embedding}\label{sec: embeddings}
OpenAI's embedding API
\subsubsection{Participant Extraction}\label{sec: participant_extraction}
Chatgpt for major participant extraction and another model for entity linking

\subsubsection{Topic Assignment}\label{sec: topic_assignment}
Chatgpt to assign topics to each cluster