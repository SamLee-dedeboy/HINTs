\section{Methodology}
\subsection{Modeling}
\subsubsection{Hypergraph Construction}
A hypergraph is a generalization of a graph in which an edge can connect more than two nodes~\cite{hypergraph_definition}.
A hyperedge thus represents a multi-way relationship between nodes.
In this paper, we model two types of hypergraphs: article hypergraph and participant hypergraph, where articles and participants are the nodes, respectively.
\textit{Participants} are the core components that the article's content discuss~\cite{use_other_works_to_refine_definition}.
For example, in a news article, the participants can be named entities such as people, organizations, or locations.
In a research article, the participants can be the concepts or techniques used in the article.

Following the definition of a hypergraph node, a hyperedge can be used to represent two types of multi-way relationships:
(1) A hyperedge between \textit{articles} can be constructed if the articles all mention the same participant. 
In this case, the hyperedge represents the co-mention of a participant, i.e.\ a named entity or a concept;
(2) A hyperedge between \textit{participants} can be constructed if the participants are mentioned together in the same article.
In this case, the hyperedge represents a co-occurrence relationship between participants.

Although these two types of hyperedges are constructed differently, we utilize the \textit{dual} of a hypergraph to simplify the construction process.
The dual of a hypergraph is simply another hypergraph, where the hyperedges are now nodes and the nodes are now hyperedges. (Add formulas here to explain).
Therefore, we first model the articles as nodes and participants as hyperedges to construct the article hypergraph $H_A$.
Then we apply a hierarchical clustering algorithm on $H_A$.
The detail of the clustering algorithm is explained in~\autoref{sec: clustering}.
The result of the clustering algorithm represents topics that are discussed in the articles.
Then, we apply the same clustering algorithm on the dual of the hypergraph, which is the participant hypergraph $H_P$.
The result represents groups of participants that are frequently mentioned together in the articles.

\subsubsection{Hierarchical Clustering}\label{sec: clustering}
We implement our hierarchical clustering algorithm following the agglomerative clustering approach.
We combine the semantic similarity $S_s$ and connectivity similarity $S_c$ to calculate the distance between two nodes.
$S_s(i, j)$ is calculated using the cosine similarity of the embeddings of the two nodes.
For article nodes, the embeddings are simply the document embeddings of the articles.
For participant nodes, the embeddings are the average of the embeddings of the articles that mention the participant.
$S_c$ is calculated using the Jaccard similarity of the two nodes' neighbors (\autoref{eq:connectivity_similarity}).
\begin{equation}\label{eq:connectivity_similarity}
    S_c(i, j) = \frac{|N_i \cap N_j|}{|N_i \cup N_j|}
\end{equation}
A weighting factor $\alpha$ is used to balance the two similarities, as shown in~\autoref{eq: similarity}.
\begin{equation}\label{eq: similarity}
    S = \alpha S_s + (1-\alpha) S_c
\end{equation}

\subsubsection{Topic Assignment}
Chatgpt to assign topics to each cluster

\subsection{Preprocessing}
The Methodology can work for any unstructured dataset
\subsubsection{Summarization}
Chatgpt for summarization
\subsubsection{Document Embedding}
OpenAI's embedding API
\subsubsection{Participant Extraction}\label{sec:participant_extraction}
Chatgpt for major participant extraction and another model for entity linking