\subsection{Design Rationale}~\label{sec: design_rationale}
HyperMap is designed for analysts to explore and reorganize a corpus for their analysis.
Our design rationale to foster interpretation is based on the design guidelines proposed by Chuang et al.~\cite{chuang2012interpretation}.
We reuse their definitions of \textit{Model Alignment}, \textit{Progressive Disclosure}, and \textit{Unit of Analysis} when describing our design rationale.
We first identify common analysis tasks from previous works.
Then, we derive our design considerations (DC) from the analysis tasks.
We take the DCs into account when making our model decisions and visualization design in~\autoref{sec: methodology}.
Finally, we explain how we achieve model alignment by applying the design considerations to our system.

\subsubsection{Analysis Tasks}
We derive our target analysis tasks from topic- and entity-based approaches.
Topic-based approaches aim to support document understanding by visualizing the topic structure of the documents.
Investigation of the topic structure seeks to answer the question: \textit{What topics are discussed in the corpus, and how are they related?}
Entity-based approaches support investigative analysis by visualizing entities and their relations.
Similarly, the investigation of the entities seeks to answer the question: \textit{What entities are involved in the corpus, and how are they related?}: 
We aim to support both tasks simultaneously as they are fundamental to subsequent tasks and intertwined in a real-world scenario.

\subsubsection{Design Considerations}
To support the aforementioned analysis tasks, we derive the following design considerations (DCs):
\begin{itemize}
  \item \textbf{DC1: Overview of topic structures and entity connections}
  Given a corpus, the topic structures and entity connections can be complex and cover a wide range of articles and entities. 
  The overview seeks to cover all the articles and entities by hiding the details.
  This sets the ground for the user to discover their targets of interest.
  \item \textbf{DC2: Progressive Disclosure}
  To facilitate investigation, it is important to support users to drill down from a high-level overview to intermediate abstractions.
  This includes disclosure of a specific topic's sub-structure, the containing articles, and the connections to entities.
  \item \textbf{DC3: Model Alignment}
  Our choice of model should align well with the analyst's mental model when conducting the analysis tasks.
  This means our model should directly operate on the units of analysis, which are topics (groups of similar articles) and entities.
  Then by properly visualizing the abstractions of the model, we are safe to produce a good model alignment.
  \item \textbf{DC4: Detailed analysis of the target of interest}
  The investigation of topic structures and entity connections often leads to a target of interest, which can be a topic or an entity.
  After such investigation, previous works usually only provide the user with a list of documents that are relevant to the target of interest.
  This is perhaps due to the lack of a unified way to analyze the target of interest under different contexts.
  The advance of LLMs presents a promising solution to this problem by transforming almost any analysis task into a question-answering task.
  We thus include this task to fill the gap in previous works.
\end{itemize}





