% \newgeometry{left = 0.1cm,top=1cm,bottom=1cm}
\clearpage
\begin{center}
    \begin{tabular}{ | p{3.0cm} | p{3.0cm} | p{13cm} | }
        \hline
        \textbf{Task} & \textbf{Dataset} & \textbf{Prompt}\\
        \hline
        Summarization & All The News & \makecell[tl]{\{\\
        \quad\textbf{Role:} `system' \\
        \quad\textbf{Content:} `You are a summarization system that summarizes events happened between the\\
        \quad main characters of a news article. The user will provide you with a news article to summarize.\\
        \quad Try to summarize the article with no more than three sentences. \\
        \quad Reply starts with `The article discussed \dots \\
        \}
        } \\ 
        \hline
        Character Extraction & All The News & \makecell[tl]{\textit{Sub-task 1: Extract event }\\ 
        \{\\ 
        \quad \textbf{Role:} `system' \\
        \quad \textbf{Content:} `You are a state-of-the-art event extraction system. Your task is to extract only the \\
        \quad most important event from news articles. \\
        \quad Strictly extract only one event. This event should be the most important event in the article. \\
        \quad Also extract the trigger that indicates the occurrence of the event. \\
        \quad The events should be human-readable. \\
        \quad Reply in this format: [\textit{event} -- \textit{trigger}] '\\ 
        \}, \\
    \{\\ 
    \quad\textbf{Role:} `user', \textbf{Content:} `This is the news article:\{\textit{sentence}\}' \\ 
    \}\\
    \hline
    \textit{Sub-task2: Extract characters} \\ 
    \{\\
    \quad\textbf{Role:} `system' \\ 
    \quad\textbf{Content:} `You are a named entity recognition model. You will be given an article and the event \\
    \quad recognized in that article by the user.  \\
    \quad The format of the input defining event in article will be: [\textit{event - category}]; \\
    \quad Extract the main characters involved in that event. The number of main characters should be 2\\
    \quad or less strictly. Ignore any other characters other than the two main characters. \\
    \quad Be as concise as possible. Reply with the following format: [\textit{character 1}],[\textit{character 2}] \dots'\\
    \},\\
    \{\\
    \quad \textbf{Role:} `system', \textbf{Name:} `example\_user' \\
    \quad \textbf{Content:} `Article: The article discussed the extensive history of doping in Russia, dating back to the\\
    \quad 1983 Soviet Union's detailed instructions to inject top athletes with anabolic steroids in order to \\ 
    \quad ensure dominance at the Los Angeles Olympics. \dots \\
    %  Dr.Sergei Portugalov, a key figure in Russia's \\
    % \quad current doping scandal, was named as the mastermind behind the doping program. The revelations of  \\
    % \quad these schemes led to the banning of Russia's track and field  team from the Rio Games, the most \\
    % \quad severe doping penalty in Olympic history.'\\
    \quad Event: [Russia's doping scandal - sports scandal]\\
    \}, \\
    \{ \\
    \quad\textbf{Role:} `system', \textbf{Name:} `example\_system', \textbf{Content:} [Russia], [Dr.Sergei Portugalov]\\
    \}, \\
    \{\\
    \quad\textbf{Role:} `user', \textbf{Content: }`Article: \{\textit{sentence}\} Event: \{\textit{event}\}' \\
    \},
    } \\
    \hline
    \makecell[tl]{Character\\ Disambiguation}  & VisPub & \makecell[tl]{\{\\
    \quad\textbf{Role:} `system'\\
    \quad\textbf{Content:} `User will provide a list of keywords from research paper abstracts.'\\ 
    \quad The input will be in the format: keyword 1, keyword 2, \dots \\
    \quad Which phrase would best describe the list of keywords? \\
    \quad The phrase should be very specific and similar to the keywords and less than 5 words. \\
    \quad If the words are too similar to each other, simply assign one of the words as the topic. \\
    \quad Strictly assign one topic to a list of keywords. \\
    \quad Reply in the format: [\textit{research topic}]; \\
    \},\\
    \{ \\
    \quad\textbf{Role:} `system', \textbf{Name:} `example\_user', \textbf{Content:} `[storytelling, data storytelling]' \\
    \}, \\
    \{ \\
    \quad\textbf{Role:} `system', \textbf{Name:} `example\_system', \textbf{Content:} `[data storytelling]' \\
    \}, \\
    \{ \\
    \quad\textbf{Role:} `user', \textbf{Content:} `[{\textit{keywords}}]' \\   
    \} \\
    } \\
    \hline
    % \textbf{Role:} user &&&
    % \textbf{Content:} &&&"Main Characters:\textbackslash n$\{arguments\}$ \textbackslash n\textbackslash n\textbackslash n Article: $\{article\}$".format&&&(arguments=arguments, article=article)\\ 
    %  \hline

\end{tabular}
\end{center}

\clearpage
% \newgeometry{left = 0.1cm,top=1cm,bottom=1cm}
\begin{center}
  \begin{tabular}{ | p{3.0cm} | p{3.0cm} | p{13cm} | }
    \hline
    \textbf{Task} & \textbf{Dataset} & \textbf{Prompt}\\
    \hline
    \makecell[tl]{Topic Label Generation \\ for documents}  & All The News & \makecell[tl]{
    \textit{{Bottom Level:}} \\
    \{\\
    \quad\textbf{Role:} `system'\\
    \quad\textbf{Content:} `You are a news article summarization system.  \\
    \quad The user will provide you with a set of summarized news articles, your job is to further \\
    \quad summarize them into one noun phrase. \\
    \quad Use words that are already in the articles, and try to use as few words as possible.\\
    \},\\
    \{\\
    \quad\textbf{Role:} `system'\\
    \quad\textbf{Name:} `example\_user'\\ 
    \quad\textbf{Content:} `Article 1: The article discussed an engineering company that creates mobile research\\
    \quad robots for the military, which released a new video showcasing the `next generation' of its\\
    \quad humanoid Atlas robot. The footage showed the robot escaping and being tormented with a hock-\\
    \quad ey stick, but it quickly recovered. In December 2013, Google bought Boston, the company be-\\
    \quad hind the robots, \dots \\ 
    \quad Article 2: The article discussed the rise of robots and artificial intelligence (AI) in various\\
    \quad industries, including Amazon's commercial delivery to a customer in Cambridge, England and\\
    \quad the testing of driverless vehicles and drones. It also mentioned the potential job losses due to \\
    \quad automation, but argued that the panic is exaggerated and that new jobs will be created. \\
    \quad Article 3: \{\textit{summary of article 3 \dots}\}\\ 
    \quad Article 4: \{\textit{summary of article 4 \dots}\}' \\
    \},\\
    \{\\
    \quad\textbf{Role:} `system', \textbf{Name:} `example\_system'\\ 
    \quad\textbf{Content:} `Robotic Advancements and Concerns'\\
    % { "role": "user", "content": summaries_message}
    \}, \\
    \{\\
    \quad\textbf{Role:} `user', \textbf{Content:} `Article 1: \{\textit{article summary 1}\}, Article 2: \{\textit{article summary 2}\} \dots'\\ 
    \} \\
    \{ \\
    \quad\textbf{Role:} `system' \\
    \quad\textbf{Content:} `You are a news article categorization system. \\
    \quad The user will provide you with a list of sub-topics of news articles and a few examples from the\\
    \quad sub-topics. Your job is to further categorize the sub-topics into a single noun phrase that best\\
    \quad summarizes all the sub-topics. Try to reuse the words in the examples.\\
    \} \\ \\
    \hline
    \textit{Intermediate Level:}\\
    \{ \\
    \quad\textbf{Role:} `system' \quad\textbf{Name:} `example\_user'\\ 
    \quad\textbf{Content:} `Sub-Topics: Increasing Gun Violence in Chicago, Crime Rates and Policing Tactics,\\
    \quad Misconceptions about Crime in the United States, Global Events and Optimism\\
    \quad Article 1: The article discussed the major events of 2016, including the Orlando nightclub shoot-\\
    \quad ing, Bastille Day attack in Nice, French priest assassination, and cafe attack in Dhaka. \dots\\
    % It also mentioned\\
    % \quad the Brexit vote, police shootings of black men, reprisal shootings of officers, and the election of \\
    % \quad Donald Trump. However, Harvard professor Steven Pinker provided a different perspective, stating\\
    % \quad that overall, the world is still in a more peaceful period than at any other time in history, with\\
    % \quad improvements in poverty, child mortality, illiteracy, and global inequality. He also expressed\\
    % \quad conditional optimism for the future.\\
    \quad Article 2: The article discussed how serious crimes in the city increased last year while the \\
    \quad number of arrests decreased \dots \\
    % The murder rate jumped 4.5 percent and robberies rose by 2 percent. \\
    % \quad Marijuana arrests declined significantly, but gun busts went up. The NYPD Deputy Commissioner \\
    % \quad for Operations mentioned that arrests have been more targeted towards a `core population that is\\
    % \quad driving crime.' Manhattan and The Bronx saw an increase in overall crime.\\
    \}, \\
    \{\\
    \quad\textbf{Role:} `system' \quad\textbf{Name:} `example\_system'\\ 
    \quad\textbf{Content:} `Crimes in the United States'\\
    \},\\
    \\
    } \\
    \hline
\end{tabular}
\end{center}

\clearpage
% \newgeometry{left = 0.1cm,top=1cm,bottom=1cm}
\begin{center}
  \begin{tabular}{ | p{3.0cm} | p{3.0cm} | p{13cm} | }
    % \hline
    % \textbf{Task} & \textbf{Dataset} & \textbf{Prompt}\\
    \hline
    % \cline{2-3}
    & VisPub & \makecell[tl]{\textit{Bottom Level:}\\
    \{\\
    \quad\textbf{Role:} `system'\\
    \quad\textbf{Content:} `You are a visualization research paper summarization system. The user will provide \\
    \quad you with a set of abstracts of visualization research papers. They are manually categorized by \\
    \quad another person, so they are discussing the same topic.  Your job is to find out what that topic is.\\
    \quad Reply with less than five words. '\\
    \},\\
    \{\\
    \quad\textbf{Role:} `system' \quad\textbf{Name:} `example\_user'\\
    \quad\textbf{Content:} `Abstract 1: Many datasets such as scientific literature collections contain multiple \\
    \quad heterogeneous facets which derive implicit relations, as well as explicit relational references \\
    \quad between data items \dots \\ 
    \quad Abstract 2: We present PivotPaths, an interactive visualization for exploring faceted information\\
    \quad resources \dots \\
    \},\\
    \{\\
    \quad\textbf{Role:} `system' \quad\textbf{Name:} `example\_system'\\
    \quad\textbf{Content:}`Faceted browsing visualization'\\
    \},\\
    \{\\
    \quad\textbf{Role:} `user'\\
    \quad\textbf{Content:} `Abstract 1: \{\textit{abstract 1}\}, Abstract 2: \{\textit{abstract 2}\} \dots'\\
    \}\\
    \\
    \hline
    \textit{Intermediate Level:}\\
    \{\\
    \quad\textbf{Role:} `system'\\
    \quad\textbf{Content:} `You are a visualization research paper summarization system. \\
    \quad You generate topics for a set of visualization research papers.\\
    \quad The user will provide you with a list of sub-topics and a few example abstracts from the sub-\\
    \quad topics. Your job is to further categorize the sub-topics into a single noun phrase that best \\
    \quad summarizes all the sub-topics. Try to reuse the words in the sub-topics.\\
    \quad Reply with a single noun phrase without any line breaks. Be as concise as possible.'\\
    \},\\
    \{\\
    \quad\textbf{Role:} `system' \quad\textbf{Name:} `example\_user'\\
    \quad\textbf{Content:} `Sub-Topics: Underwater 3D scene reconstruction from acoustic imaging sonar data, \\
    \quad Visualization of Sound Propagation in Room Acoustics, Underwater seabed visualization;\\
    \quad Abstract 1: The development of a high speed multi-frequency continuous scan sonar at Sonar\\
    \quad Research Development Ltd has resulted in the acquisition of extremely accurate, high resolution\\
    \quad bathymetric data. \dots\\
    \quad Abstract 2:  \dots '\\
    \},\\
    \{\\
    \quad\textbf{Role:} `system' \quad\textbf{Name:} `example\_system'\\
    \quad\textbf{Content:} `Visualization of Underwater Acoustic Scenes and Sound Propagation'\\
    \},\\
    \{\\
    \quad\textbf{Role:} `user'\\
    \quad\textbf{Content:} `Sub-Topics: \{\textit{sub topics}\}, Abstract 1: \{\textit{abstract 1}\}, Abstract 2: \{\textit{abstract 2}\}, \dots'\\
    \}\\
    }\\
    \hline
\end{tabular}
\end{center}

\clearpage
% \newgeometry{left = 0.1cm,top=1cm,bottom=1cm}
\begin{center}
  \begin{tabular}{ | p{3.0cm} | p{3.0cm} | p{13cm} | }
    \hline
    \textbf{Task} & \textbf{Dataset} & \textbf{Prompt}\\
    \hline
    Topic Label Generation for characters & All The News & \makecell[tl]{
    \{\\
    \quad\textbf{Role:} `system'\\
    \quad\textbf{Content:} `You are an entity summarization system.\\
    \quad The user will provide you with a list of entities, they can be people, places, or things.\\
    \quad The user wants to get a gist of what entities are in the list.\\
    \quad First, split the entities into different categories.\\
    \quad Then, assign each category a human-readable name.\\
    \quad If entities in a category are all related to a specific entity, use that entity as the category.\\
    \quad Limit the number of categories to be less than 5 by keeping only the important categories. \\
    \quad Reply with the following format: Category 1, Category 2, Category 3, \dots \\
    \quad Do not reply more than 5 categories.'\\
    \},\\
    \{\\
    \quad\textbf{Role:} `system', \textbf{Name:} `example\_user\_1'\\ 
    \quad\textbf{Content:} `Entities: Refugee, Refugee camp, Internally displaced person, Asylum seeker'\\
    \}, \\
    \{\\
    \quad\textbf{Role:} `system', \textbf{Name:} `example\_system\_1', \textbf{Content:} `Refugee'\\
    \},\\
    \{\\
    \quad\textbf{Role:} `system', \textbf{Name:} `example\_user\_2'\\ 
    \quad\textbf{Content:} `Entities: JetBlues, American Airlines, Delta Airlines, Southwest Airlines, United\\
    \quad Airlines, Alaksa Airlines'\\
    \}, \\
    \{\\
    \quad\textbf{Role:} `system', \textbf{Name:} `example\_system\_2', \textbf{Content:} `US airline companies'\\
    \},\\
    \{\\
    \quad\textbf{Role:} `user', \textbf{Content:} `Entities: \{\textit{Entity List}\}'\\
    \},\\
    } \\
    \cline{2-3}
     & VisPub & \makecell[tl]{
    \{\\
    \quad\textbf{Role:} `system'\\
    \quad\textbf{Content:} `You are a visualization research paper keyword summarization system.\\
    \quad The user will provide you with a list of keywords, they are terminologies in visualization research\\
    \quad papers. The user wants to get a gist of what keywords are in the list, but the list is too long.\\
    \quad Pick out only a few keywords that best represents the list. Avoid picking out overlapping keywords. \\
    \quad Limit the number of picked keywords to be less than 5 by keeping only the important ones. \\
    \quad Reply with the following format: Keyword\_1, Keyword\_2, Keyword\_3, \dots \\
    \},\\
    \{\\
    \quad\textbf{Role: } `user', \textbf{Content:} `Keywords: \{\textit{keyword list}\}' \\
    \}\\
    }\\
    \hline
\end{tabular}
\end{center}
\clearpage
\section{Chat Responses for usage scenario 2}
\quad \textit{Q: } Given the selected documents, what should I consider when designing text visualization?

\textit{A: }
When designing text visualization, here are some key considerations based on the selected articles:\\
When designing text visualization, you should consider the following aspects based on the selected articles: \\
1. Amount of text: Determine the appropriate amount of text to include in the visualization. The articles suggest that adding text annotations to visualizations can enhance understanding and should not be penalized. Participants preferred charts with more textual annotations over charts with fewer annotations or text alone. \\
2. Semantic content: Consider the type of information conveyed by the text. Text that describes statistical or relational components of a chart leads to more takeaways referring to statistics or relational comparisons. \\
3. Placement of text: Text should be placed strategically within the visualization. Different kinds of information are best placed in the title or closer to the data, depending on the semantic level. \\
4. Integration of text and charts: Aim for a tight coupling between text and charts to support the synthesis of information. Investigate ways to establish interactive references between text and charts, allowing users to construct connections effortlessly. \\
5. Real-time analysis: If dealing with real-time data, consider the scalability and adaptiveness of the visualization. Develop interactive systems that enable visual analysis of streaming social media data or dynamically cluster posts related to relevant topics. \\
6. Support for specific tasks: Tailor the visualization to specific tasks such as clinical chart review, argument structure analysis, or anomaly detection in social media. Incorporate insights from domain experts and conduct iterative design processes to ensure usability. \\
7. Enhancement of comprehension: Explore techniques that facilitate enhanced comprehension of complex documents by coupling text content with data visualizations. Generate on-demand visualizations based on the reader's focus within the document and evaluate their effectiveness against traditional reading experiences. \\
8. Customizability and integration with existing workflows: Provide tools that are customizable to the needs of users, such as healthcare providers and physicians, while preserving and extensively linking to the original text. Avoid obscuring or substituting the original clinical text that contains qualitative and rhetorical flavors. \\
9. Facilitating analysis of complex textual differences: Develop visual analytics environments that support the alignment of complex textual differences, prevailing in medieval text traditions or other literature domains. Provide interactive visualizations for investigating repetition, variance, and other textual hierarchies. \\
10. Evaluation and validation: Perform user studies, expert reviews, and evaluations to validate the effectiveness and utility of the visualization techniques. Collect feedback from users and iterate designs based on their needs and preferences. Consider these guidelines when designing text visualizations to ensure effective communication and comprehension of information.\\

\textit{Q: } Given the selected documents, what is considered a balanced ratio of text and charts?

\textit{A: }
Based on the selected article, it is suggested that a balanced ratio of text and charts can vary depending on the context and purpose of the visualization. 
The article mentions that heavily annotated charts were not penalized and participants preferred charts with more textual annotations. 
However, it also emphasizes the importance of considering the qualitative properties and placement of the text within the chart. 
Overall, it is recommended to use a combination of text and charts that provides additional context, guidance, and relevant information to the readers without overwhelming or distracting them from the visualizations.



